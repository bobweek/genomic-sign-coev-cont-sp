\documentclass{article}

\usepackage{arxiv}

\usepackage[utf8]{inputenc} % allow utf-8 input
\usepackage[T1]{fontenc}    % use 8-bit T1 fonts
\usepackage{lmodern}        % https://github.com/rstudio/rticles/issues/343
\usepackage{hyperref}       % hyperlinks
\usepackage{url}            % simple URL typesetting
\usepackage{booktabs}       % professional-quality tables
\usepackage{amsfonts}       % blackboard math symbols
\usepackage{nicefrac}       % compact symbols for 1/2, etc.
\usepackage{microtype}      % microtypography
\usepackage{lipsum}
\usepackage{graphicx}

\title{Spatial Scales of Local Adaptation and Host-Parasite Coevolution}

\author{
    Bob Week
   \\
    Integrative Biology \\
    Michigan State University \\
  East Lansing, MI 48824 \\
  \texttt{\href{mailto:weekrobe@msu.edu}{\nolinkurl{weekrobe@msu.edu}}} \\
   \And
    Gideon Bradburd
   \\
    Integrative Biology \\
    Michigan State University \\
  East Lansing, MI 48824 \\
  \texttt{\href{mailto:bradburd@msu.edu}{\nolinkurl{bradburd@msu.edu}}} \\
  }


% Pandoc citation processing

\usepackage{amsmath}
\usepackage{mathrsfs}
\usepackage{csquotes}
\usepackage{textcomp}


\begin{document}
\maketitle

\def\tightlist{}


\begin{abstract}
Studies of local adaptation between coevolving hosts and parasites have
profited from theory that assumes a discrete set of populations.
However, when dispersal is limited clinal patterns of phenotypic
variation may emerge. Thus, patterns of local adaptation may occur on
characteristic spatial scales determined by dispersal distances and
strengths of coevolutionary selection. Here we study a two-dimensional
continuous space model of host-parasite coevolution to understand the
relative spatial scales of phenotypic turnover and local adaptation. We
find that the species with more limited dispersal tends to be locally
adapted, but XYZ about it being ahead in the coevolutionary race. To
verify our results when model assumptions are broken, we use individual
based simulations. We find XYZ about our results when coevolution is
strong and XYZ when population densities significantly vary across
space.
\end{abstract}

\keywords{
    coevolution
   \and
    local adaptation
   \and
    continuous space
   \and
    characteristic scales
   \and
    spde
  }

\hypertarget{introduction}{%
\section{Introduction}\label{introduction}}

\begin{itemize}
\item
  It seems an old motivation for studying local adaptation in
  host-parasite coevolution comes from trying to make sense of the GMTC.
  So I think it would be good to include this.
\item
  Need to do lit review of local adaptation in coevolving host-parasite
  systems.
\item
  We should also tie in previous work on understand spatial scales of
  phenotypic variation (ie, Slatkin's 1978 ppr). Need to do lit review
  in this area too.
\end{itemize}

\hypertarget{methods}{%
\section{Methods}\label{methods}}

\hypertarget{spde-model}{%
\subsection{SPDE Model}\label{spde-model}}

\hypertarget{individual-based-simulations}{%
\subsection{Individual-Based
Simulations}\label{individual-based-simulations}}

\hypertarget{results}{%
\section{Results}\label{results}}

\hypertarget{discussion}{%
\section{Discussion}\label{discussion}}

\hypertarget{conclusion}{%
\section{Conclusion}\label{conclusion}}

\bibliographystyle{unsrt}
\bibliography{references.bib}


\end{document}
