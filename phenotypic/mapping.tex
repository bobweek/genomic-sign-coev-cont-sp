% Options for packages loaded elsewhere
\PassOptionsToPackage{unicode}{hyperref}
\PassOptionsToPackage{hyphens}{url}
%
\documentclass[
]{article}
\usepackage{lmodern}
\usepackage{amssymb,amsmath}
\usepackage{ifxetex,ifluatex}
\ifnum 0\ifxetex 1\fi\ifluatex 1\fi=0 % if pdftex
  \usepackage[T1]{fontenc}
  \usepackage[utf8]{inputenc}
  \usepackage{textcomp} % provide euro and other symbols
\else % if luatex or xetex
  \usepackage{unicode-math}
  \defaultfontfeatures{Scale=MatchLowercase}
  \defaultfontfeatures[\rmfamily]{Ligatures=TeX,Scale=1}
\fi
% Use upquote if available, for straight quotes in verbatim environments
\IfFileExists{upquote.sty}{\usepackage{upquote}}{}
\IfFileExists{microtype.sty}{% use microtype if available
  \usepackage[]{microtype}
  \UseMicrotypeSet[protrusion]{basicmath} % disable protrusion for tt fonts
}{}
\makeatletter
\@ifundefined{KOMAClassName}{% if non-KOMA class
  \IfFileExists{parskip.sty}{%
    \usepackage{parskip}
  }{% else
    \setlength{\parindent}{0pt}
    \setlength{\parskip}{6pt plus 2pt minus 1pt}}
}{% if KOMA class
  \KOMAoptions{parskip=half}}
\makeatother
\usepackage{xcolor}
\IfFileExists{xurl.sty}{\usepackage{xurl}}{} % add URL line breaks if available
\IfFileExists{bookmark.sty}{\usepackage{bookmark}}{\usepackage{hyperref}}
\hypersetup{
  pdftitle={Mapping Simulation Model Variables to SPDE Model Variables},
  hidelinks,
  pdfcreator={LaTeX via pandoc}}
\urlstyle{same} % disable monospaced font for URLs
\usepackage[margin=1in]{geometry}
\usepackage{graphicx,grffile}
\makeatletter
\def\maxwidth{\ifdim\Gin@nat@width>\linewidth\linewidth\else\Gin@nat@width\fi}
\def\maxheight{\ifdim\Gin@nat@height>\textheight\textheight\else\Gin@nat@height\fi}
\makeatother
% Scale images if necessary, so that they will not overflow the page
% margins by default, and it is still possible to overwrite the defaults
% using explicit options in \includegraphics[width, height, ...]{}
\setkeys{Gin}{width=\maxwidth,height=\maxheight,keepaspectratio}
% Set default figure placement to htbp
\makeatletter
\def\fps@figure{htbp}
\makeatother
\setlength{\emergencystretch}{3em} % prevent overfull lines
\providecommand{\tightlist}{%
  \setlength{\itemsep}{0pt}\setlength{\parskip}{0pt}}
\setcounter{secnumdepth}{-\maxdimen} % remove section numbering
\usepackage{amsmath}
\usepackage{dutchcal}
\usepackage{mathrsfs}
\usepackage{csquotes}
\usepackage{textcomp}
\usepackage[T3,T1]{fontenc}
\DeclareSymbolFont{tipa}{T3}{cmr}{m}{n}
\DeclareMathAccent{\invbreve}{\mathalpha}{tipa}{16}

\title{Mapping Simulation Model Variables to SPDE Model Variables}
\author{}
\date{\vspace{-2.5em}}

\begin{document}
\maketitle

\hypertarget{strength-of-biotic-selection-given-a-single-encounter}{%
\subsubsection{Strength of Biotic Selection Given a Single
Encounter}\label{strength-of-biotic-selection-given-a-single-encounter}}

Start by considering the component of fitness due to biotic interactions
\(W_S^{(B)}\) used in the individual-based model:

\[W_S^{(B)}(z_H,z_P)=(\iota_S)^\chi, \ \chi\sim\mathrm{Bern}\big(\pi(z_H,z_P)\big)\]

where the probability of infection is

\[\pi(z_H,z_P)=\pi_{\max}\exp\left(-\frac{\gamma}{2}(z_H-z_P)^2\right).\]

Can rewrite \(W_S^{(B)}=\exp(\chi\ln\iota_S)\) and thus
\(\mathbb E[W_S^{(B)}]\) takes the form of a moment-generating function.
Using the known mgf of a Bernoulli rv, we have

\[w_S^{(B)}=\mathbb E[W_S^{(B)}]=\big(1-\pi(z_H,z_P)\big)+\pi(z_H,z_P)e^{\ln\iota_S}=\Big(1+(\iota_S-1)\pi(z_H,z_P)\Big).\]

\[m_S(z_H,z_P)=\lim_nn(w_S^{1/n}-1)=\lim_nn\Big((w_S^{(A)})^{1/n}(w_S^{(B)})^{1/n}-1\Big)=\ln w_S^{(A)}+\ln w_S^{(B)}.\]

Assuming weak biotic selection so that \(\gamma\ll1\) we have

\[w_S^{(B)}\approx\Big(1+(\iota_S-1)\pi_{\max}\left(1-\frac{\gamma}{2}(z_H-z_P)^2\right)\Big).\]

If we further assume that \(|\iota_S-1|\ll1\), then

\[\ln w_S^{(B)}\approx\pi_{\max}(\iota_S-1)\left(1-\frac{\gamma}{2}(z_H-z_P)^2\right)=\pi_{\max}(\iota_S-1)-\frac{\gamma\pi_{\max}(\iota_S-1)}{2}(z_H-z_P)^2.\]

Under our SPDE model, the growth rate utilized took the form

\[m_S(z_H,z_P)=r_S-\frac{A_S}{2}(\theta_S-z_S)^2\pm\frac{B_S}{2}(z_H-z_P)^2,\]

where the biotic term is added for \(S=H\) and subtracted for \(S=P\).
Relating the two, we see the \(\pi_{\max}(\iota_S-1)\) term out front
can be accounted for by the intrinsic growth rate \(r_S\) leaving

\[B_S\approx\gamma\pi_{\max}|\iota_S-1|.\]

\hypertarget{parasite-biotic-selection}{%
\paragraph{Parasite Biotic Selection}\label{parasite-biotic-selection}}

Since parasites may not be close enough to a host to experience an
encounter, we need to modify our calculations. Given host density
\(\rho_H\), we can model the number of observed hosts within a region of
radius \(R\) as Poisson with parameter \(2\pi R^2\rho_H\). We can then
approximate the probability that no hosts will be within a given region
of radius \(R\) as

\[\mathbb P_R(\text{no hosts} \ | \ \rho_H)=e^{-2\pi R^2\rho_H}.\]

Then, since for infection to occur there needs to be a host present
within the interaction radius \(R_\iota\) of the parasite, the infection
probability for the parasite is replaced by

\[\pi(z_H,z_P)\to\big(1-e^{-2\pi R_\iota^2\rho_H}\big)\pi(z_H,z_P)\]

Then, repeating the same calculation as above, we find new
parameterizations of \(B_P\) and intrinsic growth rate \(r_P\):

\[B_P\approx\gamma\pi_{\max}(\iota_P-1)\big(1-e^{-2\pi R_\iota^2\rho_H}\big)\]

\[r_P^{(B)}\approx\pi_{\max}(\iota_P-1)\big(1-e^{-2\pi R_\iota^2\rho_H}\big).\]

\hypertarget{host-biotic-selection}{%
\paragraph{Host Biotic Selection}\label{host-biotic-selection}}

Since hosts may not be close enough to a parasite to experience an
encounter or may experience multiple encounters if there are several
parasites nearby (which will depend on host density), we need to modify
our calculations. Following the above approach, we can model the number
of parasites within the interaction radius \(R_\iota\) of the host as
Poisson with parameter \(2\pi R_\iota^2\rho_P\). For parasite \(i\)
within this region, we model the number of hosts within their
interaction radius \(n_i\) and compute the probability of encounter with
the focal host as the inverse of this number, \(1/n_i\). We then
accumulate the total number of encounters times their probabilities of
infection. We will denote the \(i\)th parasites trait by \(z_i\) and the
focal hosts trait by \(z_H\).

The probability of \(K\) parasites within the interaction radius of the
focal host is

\[\mathbb P(K)=\frac{1}{K!}(2\pi R_\iota^2\rho_P)^Ke^{-2\pi R_\iota^2\rho_P}.\]

For the \(i\)th parasite, the probability of \(n_i\) hosts with its
interaction radius is

\[\mathbb P(n_i)=\frac{1}{n_i!}(2\pi R_\iota^2\rho_H)^{n_i}e^{-2\pi R_\iota^2\rho_H}.\]

Denote the rv determining whether parasite \(i\) encounters the focal
host by \(\varepsilon_i\sim\mathrm{Bern}(1/n_i)\) and the probability
that parasite \(i\) infects the focal host by
\(\chi_i\sim\mathrm{Bern}(\pi(z_H,z_i))\). Then the cumulative fitness
effect of interactions with parasites on the focal host is given by

\[(\iota_H)^{\sum_{i=1}^K\chi_i\varepsilon_i}.\]

We can model infection from parasite \(i\) as a Bernoulli trial with
parameter \(\pi(z_H,z_i)/n_i\). Then, assuming these trials are
independent, the moment generating function of the sum
\(\sum_{i=1}^K\chi_i\varepsilon_i\) is just the product the moment
generating functions of each trial:

\[w_H^{(B)}=\mathbb E[(\iota_H)^{\sum_{i=1}^K\chi_i\varepsilon_i}]=\prod_{i=1}^K\Big(1+(\iota_H-1)\pi(z_H,z_i)/n_i\Big).\]

Using our small \(\gamma\) approximation again leads to

\[w_H^{(B)}\approx\prod_{i=1}^K\Big(1+(\iota_H-1)\frac{\pi_{\max}}{n_i}\left(1-\frac{\gamma}{2}(z_H-z_i)^2\right)\Big).\]

Now assuming \(|\iota_H-1|\ll1\), we have

\[\ln w_H^{(B)}\approx\sum_{i=1}^K(\iota_H-1)\frac{\pi_{\max}}{n_i}\left(1-\frac{\gamma}{2}(z_H-z_i)^2\right)=\sum_{i=1}^K(\iota_H-1)\frac{\pi_{\max}}{n_i}-(\iota_H-1)\frac{\pi_{\max}}{n_i}\frac{\gamma}{2}(z_H-z_i)^2.\]

Setting \(\bar z_P, v_P\) as the local parasite mean trait and trait
variance respectively, we further approximate with

\[\ln w_H^{(B)}\approx K\left((\iota_H-1)\frac{\pi_{\max}}{n_i}-(\iota_H-1)\frac{\pi_{\max}}{n_i}\frac{\gamma}{2}[(z_H-\bar z_P)^2+v_P]\right).\]

Finally, we approximate \(K\) and the \(n_i\) with their respective
expectations \(2\pi R_\iota^2\rho_P\) and \(2\pi R_\iota^2\rho_H\).
Hence, the updated strength of biotic selection and biotic contribution
to intrinsic growth rate are

\[B_H\approx 2\pi R_\iota^2\rho_P\gamma(\iota_H-1)\frac{\pi_{\max}}{2\pi R_\iota^2\rho_H}=\gamma\pi_{\max}(\iota_H-1)\frac{\rho_P}{\rho_H},\]

\[r_H^{(B)}\approx\pi_{\max}(\iota_H-1)(1-\gamma v_P/2)\frac{\rho_P}{\rho_H}.\]

\hypertarget{additive-genetic-variance}{%
\subsubsection{Additive Genetic
Variance}\label{additive-genetic-variance}}

At equilibrium and under weak coevolution, we have the approximation

\[G_P\approx \sqrt{\mu_P/(A_P+B_P)}\]

\[G_H\approx \sqrt{\mu_H/(A_H-B_H)}\]

\begin{itemize}
\tightlist
\item
  These \(G\)'s are measured at some local scale. Perhaps measured by
  dispersal distance \(\sigma\)?
\end{itemize}

\hypertarget{expressed-phenotypic-variation}{%
\subsubsection{Expressed Phenotypic
Variation}\label{expressed-phenotypic-variation}}

This should always be approximately true, especially for large
population sizes

\[v_S\approx G_S+E_S.\]

However, which \(v\) and which \(G\) depends on the scale considered.
The \enquote{local population scale} may be determined by dispersal
distance, but interactions occur at a particular radius. So when
averaging over potential interaction partners, the appropriate \(v\) and
\(G\) should be measured at the scale of the interaction radius.

\hypertarget{population-density}{%
\subsubsection{Population Density}\label{population-density}}

Given \(n\) individuals within a radius \(R_S\), the fitness of an
individual in species \(S\) will be attenuated by \((\kappa_S)^n\). This
corresponds to an additive effect on growth rate of \(-n\ln\kappa_S\).
Then, at the scale on which competition occurs the equilibrium
expectation is

\[\rho_P\approx \frac{-1}{\ln\kappa_P}\Big(r_P-\frac{1}{2}\sqrt{\mu_P(A_P+B_P)}\Big)\]

\[\rho_H\approx \frac{-1}{\ln\kappa_H}\Big(r_H-\frac{1}{2}\sqrt{\mu_S(A_H-B_H)}\Big)\]

where

\[r_H=\ln\alpha_H+\pi_{\max}(\iota_S-1)(1-\gamma v_P/2)\frac{\rho_P}{\rho_H},\]

\[r_P=\ln\alpha_P+\pi_{\max}(\iota_P-1)\big(1-e^{-2\pi R_\iota^2\rho_H}\big),\]

\[B_H\approx\gamma\pi_{\max}(1-\iota_H)\frac{\rho_P}{\rho_H}\]

\[B_P\approx\gamma\pi_{\max}(\iota_P-1)\big(1-e^{-2\pi R_\iota^2\rho_H}\big)\]

\begin{itemize}
\tightlist
\item
  Need to test these predictions against simulations\ldots{}
\end{itemize}

\begin{verbatim}
##          Parameter Expectation Observation
## 1     host density   12.888329        11.7
## 2 parasite density    8.707538         8.2
\end{verbatim}

\end{document}
