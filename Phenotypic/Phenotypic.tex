\documentclass{article}

\usepackage{arxiv}

\usepackage[utf8]{inputenc} % allow utf-8 input
\usepackage[T1]{fontenc}    % use 8-bit T1 fonts
\usepackage{lmodern}        % https://github.com/rstudio/rticles/issues/343
\usepackage{hyperref}       % hyperlinks
\usepackage{url}            % simple URL typesetting
\usepackage{booktabs}       % professional-quality tables
\usepackage{amsfonts}       % blackboard math symbols
\usepackage{nicefrac}       % compact symbols for 1/2, etc.
\usepackage{microtype}      % microtypography
\usepackage{lipsum}
\usepackage{graphicx}

\title{The Phenotypic Signature of Host-Parasite Coevolution in Continuous
Space}

\author{
    Bob Week
   \\
    Integrative Biology \\
    Michigan State University \\
  East Lansing, MI 48824 \\
  \texttt{\href{mailto:weekrobe@msu.edu}{\nolinkurl{weekrobe@msu.edu}}} \\
   \And
    Gideon Bradburd
   \\
    Integrative Biology \\
    Michigan State University \\
  East Lansing, MI 48824 \\
  \texttt{\href{mailto:bradburd@msu.edu}{\nolinkurl{bradburd@msu.edu}}} \\
  }


% Pandoc citation processing

\usepackage{amsmath}


\begin{document}
\maketitle

\def\tightlist{}


\begin{abstract}
Here we identify the phenotypic signature of host-parasite coevolution
in continuous space
\end{abstract}

\keywords{
    blah
   \and
    blee
   \and
    bloo
   \and
    these are optional and can be removed
  }

\hypertarget{introduction}{%
\section{Introduction}\label{introduction}}

The host is denoted \(H\) and the parasite \(P\). Spatial location is
\(x\) (\(\in\mathbb R^1\) or \(\mathbb R^2\)) and individual trait value
of a host/parasite is \(z_H/z_P\in\mathbb R^1\). Assuming a
trait-matching model, where host fitness is minimized and parasite
fitness is maximized when \(z_H=z_P\), we have the following two fitness
functions:

\begin{subequations}
  \begin{equation}
    W_H\propto \exp\left(\frac{B_H}{2}(z_H-z_P)^2\right),
  \end{equation}
  \begin{equation}
    W_P\propto \exp\left(-\frac{B_P}{2}(z_H-z_P)^2\right),
  \end{equation}
\end{subequations}

where \(B_H,B_P>0\) capture the strengths of biotic selection
experienced by each species. Using a standard approach, we derive the
following non-spatial coevolutionary model:

\begin{subequations}
  \begin{equation}
    \frac{d\bar z_H}{dt}=-G_HB_H(\bar z_P-\bar z_H),
  \end{equation}
  \begin{equation}
    \frac{d\bar z_P}{dt}=G_PB_P(\bar z_H-\bar z_P),
  \end{equation}
\end{subequations}

where \(G_H,G_P>0\) denote the additive genetic variances of each
species. If we assume interactions occur locally (individuals interact
only when they `collide'), then we obtain the following continuous space
model:

\begin{subequations}
  \begin{equation}
    \frac{\partial\bar z_H}{\partial t}=-G_HB_H(\bar z_P-\bar z_H)+\frac{D_H}{2}\Delta\bar z_H,
  \end{equation}
  \begin{equation}
    \frac{\partial\bar z_P}{\partial t}=G_PB_P(\bar z_H-\bar z_P)+\frac{D_P}{2}\Delta\bar z_P,
  \end{equation}
\end{subequations}

where \(D_H,D_P>0\) are dispersal rates for each species and
\(\Delta=\frac{\partial^2}{\partial x_1^2}\) in one-dimensional space
and
\(\Delta=\frac{\partial^2}{\partial x_1^2}+\frac{\partial^2}{\partial x_2^2}\)
in two-dimensional space.

\hypertarget{headings-first-level}{%
\section{Headings: first level}\label{headings-first-level}}

\label{sec:headings}

\lipsum[4] See Section \ref{sec:headings}.

\hypertarget{headings-second-level}{%
\subsection{Headings: second level}\label{headings-second-level}}

\lipsum[5]

\begin{equation}
\xi _{ij}(t)=P(x_{t}=i,x_{t+1}=j|y,v,w;\theta)= {\frac {\alpha _{i}(t)a^{w_t}_{ij}\beta _{j}(t+1)b^{v_{t+1}}_{j}(y_{t+1})}{\sum _{i=1}^{N} \sum _{j=1}^{N} \alpha _{i}(t)a^{w_t}_{ij}\beta _{j}(t+1)b^{v_{t+1}}_{j}(y_{t+1})}}
\end{equation}

\hypertarget{headings-third-level}{%
\subsubsection{Headings: third level}\label{headings-third-level}}

\lipsum[6]

\paragraph{Paragraph}
\lipsum[7]

\hypertarget{examples-of-citations-figures-tables-references}{%
\section{Examples of citations, figures, tables,
references}\label{examples-of-citations-figures-tables-references}}

\label{sec:others}

\lipsum[8] some text (Kour and Saabne 2014b, 2014a) and see Hadash et
al. (2018).

The documentation for \verb+natbib+ may be found at

\begin{center}
  \url{http://mirrors.ctan.org/macros/latex/contrib/natbib/natnotes.pdf}
\end{center}

Of note is the command \verb+\citet+, which produces citations
appropriate for use in inline text. For example,

\begin{verbatim}
   \citet{hasselmo} investigated\dots
\end{verbatim}

produces

\begin{quote}
  Hasselmo, et al.\ (1995) investigated\dots
\end{quote}

\begin{center}
  \url{https://www.ctan.org/pkg/booktabs}
\end{center}

\hypertarget{figures}{%
\subsection{Figures}\label{figures}}

\lipsum[10]

See Figure \ref{fig:fig1}. Here is how you add footnotes. {[}\^{}Sample
of the first footnote.{]}

\lipsum[11]

\begin{figure}
  \centering
  \fbox{\rule[-.5cm]{4cm}{4cm} \rule[-.5cm]{4cm}{0cm}}
  \caption{Sample figure caption.}
  \label{fig:fig1}
\end{figure}

\hypertarget{tables}{%
\subsection{Tables}\label{tables}}

\lipsum[12]

See awesome Table\textasciitilde{}\ref{tab:table}.

\begin{table}
 \caption{Sample table title}
  \centering
  \begin{tabular}{lll}
    \toprule
    \multicolumn{2}{c}{Part}                   \\
    \cmidrule(r){1-2}
    Name     & Description     & Size ($\mu$m) \\
    \midrule
    Dendrite & Input terminal  & $\sim$100     \\
    Axon     & Output terminal & $\sim$10      \\
    Soma     & Cell body       & up to $10^6$  \\
    \bottomrule
  \end{tabular}
  \label{tab:table}
\end{table}

\hypertarget{lists}{%
\subsection{Lists}\label{lists}}

\begin{itemize}
\tightlist
\item
  Lorem ipsum dolor sit amet
\item
  consectetur adipiscing elit.
\item
  Aliquam dignissim blandit est, in dictum tortor gravida eget. In ac
  rutrum magna.
\end{itemize}

\hypertarget{refs}{}
\leavevmode\hypertarget{ref-hadash2018estimate}{}%
Hadash, Guy, Einat Kermany, Boaz Carmeli, Ofer Lavi, George Kour, and
Alon Jacovi. 2018. ``Estimate and Replace: A Novel Approach to
Integrating Deep Neural Networks with Existing Applications.''
\emph{arXiv Preprint arXiv:1804.09028}.

\leavevmode\hypertarget{ref-kour2014fast}{}%
Kour, George, and Raid Saabne. 2014a. ``Fast Classification of
Handwritten on-Line Arabic Characters.'' In \emph{Soft Computing and
Pattern Recognition (Socpar), 2014 6th International Conference of},
312--18. IEEE.

\leavevmode\hypertarget{ref-kour2014real}{}%
---------. 2014b. ``Real-Time Segmentation of on-Line Handwritten Arabic
Script.'' In \emph{Frontiers in Handwriting Recognition (Icfhr), 2014
14th International Conference on}, 417--22. IEEE.

\bibliographystyle{unsrt}
\bibliography{references.bib}


\end{document}
