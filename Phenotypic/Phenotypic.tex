\documentclass{article}

\usepackage{arxiv}

\usepackage[utf8]{inputenc} % allow utf-8 input
\usepackage[T1]{fontenc}    % use 8-bit T1 fonts
\usepackage{lmodern}        % https://github.com/rstudio/rticles/issues/343
\usepackage{hyperref}       % hyperlinks
\usepackage{url}            % simple URL typesetting
\usepackage{booktabs}       % professional-quality tables
\usepackage{amsfonts}       % blackboard math symbols
\usepackage{nicefrac}       % compact symbols for 1/2, etc.
\usepackage{microtype}      % microtypography
\usepackage{lipsum}
\usepackage{graphicx}

\title{The Phenotypic Signature of Host-Parasite Coevolution in Continuous
Space}

\author{
    Bob Week
   \\
    Integrative Biology \\
    Michigan State University \\
  East Lansing, MI 48824 \\
  \texttt{\href{mailto:weekrobe@msu.edu}{\nolinkurl{weekrobe@msu.edu}}} \\
   \And
    Gideon Bradburd
   \\
    Integrative Biology \\
    Michigan State University \\
  East Lansing, MI 48824 \\
  \texttt{\href{mailto:bradburd@msu.edu}{\nolinkurl{bradburd@msu.edu}}} \\
  }


% Pandoc citation processing

\usepackage{amsmath}


\begin{document}
\maketitle

\def\tightlist{}


\begin{abstract}
Here we identify the phenotypic signature of host-parasite coevolution
in continuous space.
\end{abstract}

\keywords{
    blah
   \and
    blee
   \and
    bloo
   \and
    these are optional and can be removed
  }

\hypertarget{introduction}{%
\section{Introduction}\label{introduction}}

Space is hypothesized to play a fundamental role in the coevolutionary
process. In particular, the combined effects of local co-adaptation and
spatial movement are thought to produce complex spatial patterns of
matched and mismatched traits. However, theoretical studies
investigating the spatial signal of coevolution have for the most part
restricted themselves to understanding patterns of spatial correlation
using spatially implicit models. Hence, the fine-grained details that
can be produced in spatially explicit settings have yet to be
understood. Here we close this gap in the context of host-parasite
coevolution using continuous-space quantitative genetic models. In
particular, we study the autocorrelations of mean traits within species
and the cross-correlation of mean traits between species when trait
dynamics are driven by (1) random genetic drift, gene-flow and spatially
homogeneous abiotic stabilizing selection, (2) random genetic drift,
gene-flow, spatially homogeneous stabilizing selection and unilateral
adaptation of the host species, (3) random genetic drift, gene-flow and
unilateral adaptation of the parasite species and (4) random genetic
drift, gene-flow, stabilizing selection and host-parasite coevolution.
By comparing patterns of spatial autocorrelation and cross-correlation
in these four scenarios we are able to elucidate the signature of
coevolution in comparison to underlying non-coevolutionary processes.

\hypertarget{the-model}{%
\section{The Model}\label{the-model}}

\hypertarget{fitness}{%
\subsection{Fitness}\label{fitness}}

The host is denoted \(H\) and the parasite \(P\). Individual trait
values of hosts and parasites are respectively denoted by
\(z_H,z_P\in\mathbb R\). Assuming a trait-matching model, where host
fitness is minimized and parasite fitness is maximized when \(z_H=z_P\),
we have the following two fitness functions:

\begin{subequations}\label{fit}
  \begin{equation}
    w_H(z_H,z_P)\propto \exp\left(-\frac{A_H}{2}(\theta_H-z_H)^2+\frac{B_H}{2}(z_P-z_H)^2\right),
  \end{equation}
  \begin{equation}
    w_P(z_P,z_H)\propto \exp\left(-\frac{A_P}{2}(\theta_P-z_P)^2-\frac{B_P}{2}(z_H-z_P)^2\right),
  \end{equation}
\end{subequations}

where \(A_H,A_P>0\) capture the strengths of abiotic stabilizing
selection, \(\theta_H,\theta_P\in\mathbb{R}\) are the abiotic optimal
phenotypes and \(B_H,B_P>0\) capture the strengths of biotic selection
experienced by each species.

\hypertarget{non-spatial-dynamics}{%
\subsection{Non-spatial Dynamics}\label{non-spatial-dynamics}}

Using a standard approach, we use the above fitness functions to derive
the following non-spatial coevolutionary model:

\begin{subequations}\label{non-spatial}
  \begin{equation}
    \frac{d\bar z_H}{dt}=G_HA_H(\theta_H-\bar z_H)-G_HB_H(\bar z_P-\bar z_H),
  \end{equation}
  \begin{equation}
    \frac{d\bar z_P}{dt}=G_PA_P(\theta_P-\bar z_P)+G_PB_P(\bar z_H-\bar z_P),
  \end{equation}
\end{subequations}

where \(G_H,G_P>0\) denote the additive genetic variances of each
species. In this linear set of equations, when abiotic stabilizing
selection is absent (\(A_H=A_P=0\)), the parasite mean trait
\(\bar z_P\) evolves to match the host mean trait \(\bar z_H\).
Simultaneously \(\bar z_H\) evolves away from \(\bar z_P\). In
particular, when \(\bar z_H>\bar z_P\), the host mean trait will evolve
upwards. Similarly, when \(\bar z_H<\bar z_P\), the host mean trait will
evolve downwards. This and related models have been thoroughly studied
by Sergey Gavrilets and others. For the case when \(A_H=A_P=0\) two
possible outcomes have been demonstrated. First, when \(G_PB_P>G_HB_H\)
so that the parasite evolves faster than the host, the system
asymptotically evolves to the equilibrium \(\bar z_H=\bar z_P\). Second,
when \(G_PB_P<G_HB_H\) so that the host evolves faster than the
parasite, there is no stable equilibrium. This implies the host evolves
to escape the interaction. In the case when \(A_H,A_P\neq0\) limit
cycles can emerge.

\hypertarget{space}{%
\subsection{Space}\label{space}}

In this case mean traits become functions of spatial variables (eg.,
\(\bar z_H(x),\bar z_P(x)\) for \(x\in\mathbb R\) and
\(\bar z_H(x_1,x_2),\bar z_P(x_1,x_2)\) for
\((x_1,x_2)\in\mathbb R^2\)). As a first step, we assume all model
parameters are spatially homogeneous. If we assume interactions occur
locally (individuals interact only when they `collide') and abundance
densities are spatially homogeneous, then we can obtain the following
continuous space model:

\begin{subequations}\label{deterministic}
  \begin{equation}
    \frac{\partial\bar z_H}{\partial t}=G_HA_H(\theta_H-\bar z_H)-G_HB_H(\bar z_P-\bar z_H)+\frac{D_H}{2}\Delta\bar z_H,
  \end{equation}
  \begin{equation}
    \frac{\partial\bar z_P}{\partial t}=G_PA_P(\theta_P-\bar z_P)+G_PB_P(\bar z_H-\bar z_P)+\frac{D_P}{2}\Delta\bar z_P,
  \end{equation}
\end{subequations}

where \(D_H,D_P>0\) are dispersal rates for each species and \(\Delta\)
is the diffusion operator representing spatial movement. In particular,
\(\Delta=\frac{\partial^2}{\partial x^2}\) in one-dimensional space and
\(\Delta=\frac{\partial^2}{\partial x_1^2}+\frac{\partial^2}{\partial x_2^2}\)
in two-dimensional space.

\hypertarget{random-genetic-drift}{%
\subsection{Random Genetic Drift}\label{random-genetic-drift}}

To incorporate the effects of random genetic drift, we should be able to
justify the following SPDE model:

\begin{subequations}\label{spde}
  \begin{equation}
    \frac{\partial\bar z_H}{\partial t}=G_HA_H(\theta_H-\bar z_H)-G_HB_H(\bar z_P-\bar z_H)+\frac{D_H}{2}\Delta\bar z_H+\sqrt\frac{G_H}{N_H}\dot W_H,
  \end{equation}
  \begin{equation}
    \frac{\partial\bar z_P}{\partial t}=G_PA_P(\theta_P-\bar z_P)+G_PB_P(\bar z_H-\bar z_P)+\frac{D_P}{2}\Delta\bar z_P+\sqrt\frac{G_P}{N_P}\dot W_P,
  \end{equation}
\end{subequations}

where \(N_P,N_H\) are the local abundance densities (here assumed to be
constant in space and time) and \(\dot W_H,\dot W_P\) are space-time
white noise processes. This SPDE model may possess some major benefits
over more traditional continuous-space models of population genetics. In
particular, it is a simple extension of the stochastic heat equation,
which is known to admit solutions in dimensions one, two and three.

\hypertarget{methods}{%
\section{Methods}\label{methods}}

To study the continuous-space signature of coevolution, we can start
with an analytical approach. However, if the corresponding calculations
prove to be analytically intractable or time costly, we can employ
simulation-based methods to describe the shape of the autocorrelations
and cross-correlation in relevant regions of parameters space. The
analytical approach would begin by identifying stable attractors of the
deterministic system using classical PDE theory. Following this, we may
possibly apply theory of SPDE to describe stationary distributions
associated with each attractor. This may lend analytical solutions for
the expectation and variance of intraspecific autocorrelations and the
interspecific cross-correlation, which are defined in Appendix
\ref{corr-defs} below.

\newpage

\appendix

\hypertarget{definitions-of-spatial-autocorrelation-and-cross-correlation}{%
\section{\texorpdfstring{Definitions of Spatial Autocorrelation and
Cross-correlation
\label{corr-defs}}{Definitions of Spatial Autocorrelation and Cross-correlation }}\label{definitions-of-spatial-autocorrelation-and-cross-correlation}}

\hypertarget{autocorrelation}{%
\subsection{Autocorrelation}\label{autocorrelation}}

In one-dimensional continuous space, the autocorrelation of a function
\(f\) is defined by

\begin{equation}
  \rho_f(x)=\int_{-\infty}^{+\infty}f(y)f(y-x)dy.
\end{equation}

Sometimes the autocorrelation \(\rho_f(x)\) is written as
\((f\star f)(x)\). To extend this definition to two-dimensional space,
we can calculate a bivariate autocorrelation function of \(f\),

\begin{equation}
  \rho_f(x_1,x_2)=\int_{-\infty}^{+\infty}\int_{-\infty}^{+\infty}f(y_1,y_2)f(y_1-x_2,y_2-x_2)dy_1dy_2,
\end{equation}

and then average this quantity across distances
\(r=\sqrt{x_1^2+x_2^2}\):

\begin{equation}
  \bar\rho_f(r)=\frac{1}{2\pi r}\int_0^{2\pi}\rho_f(r\cos\theta,r\sin\theta)d\theta.
\end{equation}

\hypertarget{cross-correlation}{%
\subsection{Cross-correlation}\label{cross-correlation}}

The cross-correlation of two functions \(f,g\) is defined by

\begin{equation}
  \rho_{fg}(x)=\int_{-\infty}^{+\infty}f(y)g(y-x)dy.
\end{equation}

Sometimes this is written as \(f\star g\). Just as with the bivariate
autocorrelation above, we can extend this definition to two-dimensional
space. In particular, the bivariate cross-correlation of two functions
\(f,g\) can be written

\begin{equation}
  \rho_{fg}(x_1,x_2)=\int_{-\infty}^{+\infty}\int_{-\infty}^{+\infty}f(y_1,y_2)g(y_1-x_2,y_2-x_2)dy_1dy_2.
\end{equation}

We can also reduce this quantity to a function of distance using the
average

\begin{equation}
  \bar\rho_{fg}(r)=\frac{1}{2\pi r}\int_0^{2\pi}\rho_{fg}(r\cos\theta,r\sin\theta)d\theta.
\end{equation}

\hypertarget{a-stable-equilibrium-of-the-deterministic-system}{%
\section{A Stable Equilibrium of the Deterministic
System}\label{a-stable-equilibrium-of-the-deterministic-system}}

To find an equilibrium of the deterministic PDE (\ref{deterministic}) we
set the time derivatives of \(\bar z_H\) and \(\bar z_P\) equal to zero.
This leads to following systems of ODE's

\begin{subequations}
  \begin{equation}
    0=G_HA_H(\theta_H-\bar z_H)-G_HB_H(\bar z_P-\bar z_H)+\frac{D_H}{2}\Delta\bar z_H,
  \end{equation}
  \begin{equation}
    0=G_PA_P(\theta_P-\bar z_P)+G_PB_P(\bar z_H-\bar z_P)+\frac{D_P}{2}\Delta\bar z_P.
  \end{equation}
\end{subequations}

Solving this system for the spatial functions \(\bar z_H(x_1,x_2)\) and
\(\bar z_P(x_1,x_2)\) then returns equilibrium solutions to PDE
(\ref{deterministic}). However, we can start simple by considering
spatially homogeneous solutions. In these cases the spatial derivatives
will return zero (in particular, \(\Delta\bar z_H=\Delta\bar z_P=0\)).
Finding these solutions amounts to solve the following system of
algebraic equations

\begin{subequations}
  \begin{equation}
    0=G_HA_H(\theta_H-\bar z_H)-G_HB_H(\bar z_P-\bar z_H),
  \end{equation}
  \begin{equation}
    0=G_PA_P(\theta_P-\bar z_P)+G_PB_P(\bar z_H-\bar z_P).
  \end{equation}
\end{subequations}

This linear system is uniquely solved by

\begin{subequations}\label{non-spatial-eq}
  \begin{equation}
    \bar z_P(x_1,x_2)\equiv\frac{(A_H-B_H)A_P\theta_P+B_PA_H\theta_H}{(A_H-B_H)A_P+B_PA_H},
  \end{equation}
  \begin{equation}
    \bar z_H(x_1,x_2)\equiv\frac{(A_P+B_P)A_H\theta_H-B_HA_P\theta_P}{(A_P+B_P)A_H-B_HA_P},
  \end{equation}
\end{subequations}

where the symbol \(\equiv\) is used in place of \(=\) to emphasize these
functions are constant across space.

To understand whether this equilibrium is stable we perform stability
analysis on the non-spatial model. Notice that system
(\ref{non-spatial}) can be rewritten in matrix notation as
\(\vec{\bar z}=M\vec{\bar z}+\vec b\) where

\begin{subequations}
  \begin{equation}
    \vec{\bar z}=\left(\begin{matrix}
      \bar z_H \\ \bar z_P
    \end{matrix}\right), \ 
    \vec b =\left(\begin{matrix}
      G_HA_H\theta_H \\ G_PA_P\theta_P
    \end{matrix}\right),
  \end{equation}
  \begin{equation}
    M=\left(\begin{matrix}
      G_H(B_H-A_H) & -G_HB_H \\
      G_PB_P & -G_P(A_P+B_P)
    \end{matrix}\right).
  \end{equation}
\end{subequations}

It turns that if the real parts of the eigenvalues of \(M\) are negative
then equilibrium (\ref{non-spatial-eq}) is stable. This condition holds
in general when \(B_H<A_H+(A_P+B_P)G_P/G_H\) and \(B_H<A_H(1+B_P/A_P)\).
A sufficient condition with clear biological intuition is \(B_H<A_H\),
which just means the strength of biotic selection on the host is less
than the strength of abiotic stabilizing selection.

\hypertarget{stationary-solutions-of-spde}{%
\section{\texorpdfstring{Stationary Solutions of SPDE
(\ref{spde})}{Stationary Solutions of SPDE ()}}\label{stationary-solutions-of-spde}}

\bibliographystyle{unsrt}
\bibliography{references.bib}


\end{document}
