\documentclass{article}

\usepackage{arxiv}

\usepackage[utf8]{inputenc} % allow utf-8 input
\usepackage[T1]{fontenc}    % use 8-bit T1 fonts
\usepackage{lmodern}        % https://github.com/rstudio/rticles/issues/343
\usepackage{hyperref}       % hyperlinks
\usepackage{url}            % simple URL typesetting
\usepackage{booktabs}       % professional-quality tables
\usepackage{amsfonts}       % blackboard math symbols
\usepackage{nicefrac}       % compact symbols for 1/2, etc.
\usepackage{microtype}      % microtypography
\usepackage{lipsum}
\usepackage{graphicx}

\title{The Phenotypic Signature of Host-Parasite Coevolution in Continuous
Space}

\author{
    Bob Week
   \\
    Integrative Biology \\
    Michigan State University \\
  East Lansing, MI 48824 \\
  \texttt{\href{mailto:weekrobe@msu.edu}{\nolinkurl{weekrobe@msu.edu}}} \\
   \And
    Gideon Bradburd
   \\
    Integrative Biology \\
    Michigan State University \\
  East Lansing, MI 48824 \\
  \texttt{\href{mailto:bradburd@msu.edu}{\nolinkurl{bradburd@msu.edu}}} \\
  }


% Pandoc citation processing

\usepackage{amsmath}
\usepackage{mathrsfs}
\usepackage{csquotes}
\usepackage{textcomp}


\begin{document}
\maketitle

\def\tightlist{}


\begin{abstract}
Here we identify the phenotypic signature of host-parasite coevolution
in continuous space.
\end{abstract}

\keywords{
    blah
   \and
    blee
   \and
    bloo
   \and
    these are optional and can be removed
  }

\hypertarget{introduction}{%
\section{Introduction}\label{introduction}}

Space is hypothesized to play a fundamental role in the coevolutionary
process. In particular, the combined effects of local co-adaptation and
spatial movement are thought to produce complex spatial patterns of
matched and mismatched traits. However, theoretical studies
investigating the spatial signal of coevolution have for the most part
restricted themselves to understanding patterns of spatial correlation
using spatially implicit models. Hence, the fine-grained details that
can be produced in spatially explicit settings have yet to be
understood. Here we close this gap in the context of host-parasite
coevolution using continuous-space models inspired by classical
quantitative genetic theory. In particular, we study the
autocorrelations of mean traits within species and the cross-correlation
of mean traits between species when trait dynamics are driven by (1)
random genetic drift, gene-flow and abiotic stabilizing selection, (2)
drift, gene-flow, abiotic stabilizing selection and unilateral
adaptation of the host species, (3) drift, gene-flow, stabilizing
selection and unilateral adaptation of the parasite species and (4)
drift, gene-flow, stabilizing selection and host-parasite coevolution.
In each case we assuming abiotic stabilizing selection is spatially
homogeneous. By comparing patterns of spatial autocorrelation and
cross-correlation in these four scenarios we are able to elucidate the
signature of coevolution.

\hypertarget{questions}{%
\section{Questions}\label{questions}}

\begin{itemize}
\tightlist
\item
  What are the characteristic spatial scales of local adaptation in each
  species?

  \begin{itemize}
  \tightlist
  \item
    How do these depend on rates/distances of dispersal and strengths of
    selection?
  \end{itemize}
\item
  What is the characteristic spatial scale at which coevolution becomes
  visible?

  \begin{itemize}
  \tightlist
  \item
    How does this depend on dispersal distances, strengths of selection,
    local effective sizes and local additive genetic variation?
  \end{itemize}
\item
  What are the characteristic scales of spatial autocorrelation and
  cross-correlation?

  \begin{itemize}
  \tightlist
  \item
    Does this answer both questions above?
  \end{itemize}
\item
  How do these answers change when selection, local effective sizes or
  local additive genetic variances are spatially heterogeneous?

  \begin{itemize}
  \tightlist
  \item
    What if these parameters follow deterministic clines? Linear or
    quadratic?
  \item
    What if they follow stochastic patterns such as Gaussian fields?

    \begin{itemize}
    \tightlist
    \item
      In particular, consider the characteristic scale of environmental
      heterogeneity.
    \end{itemize}
  \item
    In these scenarios, how does autocorrelation and cross-correlation
    change across space? What is the characteristic scale at which their
    characteristic scales change?
  \end{itemize}
\item
  Considering answers to all of the above, what is the best sampling
  scheme for detecting coevolution?

  \begin{itemize}
  \tightlist
  \item
    More intensive sampling at a few locations?
  \item
    Less intensive sampling at many locations?
  \end{itemize}
\item
  Future directions:

  \begin{itemize}
  \tightlist
  \item
    Does the interplay between coevolution and abundance dynamics lead
    to patchy distributions?
  \end{itemize}
\end{itemize}

\hypertarget{the-model}{%
\section{The Model}\label{the-model}}

\hypertarget{fitness}{%
\subsection{Fitness}\label{fitness}}

The host is denoted \(H\) and the parasite \(P\). Individual trait
values of hosts and parasites are respectively denoted by
\(z_H,z_P\in\mathbb R\). Assuming a trait-matching model, where host
fitness is minimized and parasite fitness is maximized when \(z_H=z_P\),
we have the following two fitness functions:

\begin{subequations}\label{fit}
  \begin{equation}
    w_H(z_H,z_P)\propto \exp\left(-\frac{A_H}{2}(\theta_H-z_H)^2+\frac{B_H}{2}(z_P-z_H)^2\right),
  \end{equation}
  \begin{equation}
    w_P(z_P,z_H)\propto \exp\left(-\frac{A_P}{2}(\theta_P-z_P)^2-\frac{B_P}{2}(z_H-z_P)^2\right),
  \end{equation}
\end{subequations}

where \(A_H,A_P>0\) capture the strengths of abiotic stabilizing
selection, \(\theta_H,\theta_P\in\mathbb{R}\) are the abiotic optimal
phenotypes and \(B_H,B_P>0\) capture the strengths of biotic selection
experienced by each species.

\hypertarget{non-spatial-dynamics}{%
\subsection{Non-spatial Dynamics}\label{non-spatial-dynamics}}

Using a standard approach, we use the above fitness functions to derive
the following non-spatial coevolutionary model:

\begin{subequations}\label{non-spatial}
  \begin{equation}
    \frac{d\bar z_H}{dt}=G_HA_H(\theta_H-\bar z_H)-G_HB_H(\bar z_P-\bar z_H),
  \end{equation}
  \begin{equation}
    \frac{d\bar z_P}{dt}=G_PA_P(\theta_P-\bar z_P)+G_PB_P(\bar z_H-\bar z_P),
  \end{equation}
\end{subequations}

where \(G_H,G_P>0\) denote the additive genetic variances of each
species. In this linear set of equations, when abiotic stabilizing
selection is absent (\(A_H=A_P=0\)), the parasite mean trait
\(\bar z_P\) evolves to match the host mean trait \(\bar z_H\).
Simultaneously \(\bar z_H\) evolves away from \(\bar z_P\). In
particular, when \(\bar z_H>\bar z_P\), the host mean trait will evolve
upwards. Similarly, when \(\bar z_H<\bar z_P\), the host mean trait will
evolve downwards. This and related models have been thoroughly studied
by Sergey Gavrilets and others. For the case when \(A_H=A_P=0\) two
possible outcomes have been demonstrated. First, when \(G_PB_P>G_HB_H\)
so that the parasite evolves faster than the host, the system
asymptotically evolves to the equilibrium \(\bar z_H=\bar z_P\). Second,
when \(G_PB_P<G_HB_H\) so that the host evolves faster than the
parasite, there is no stable equilibrium. This implies the host evolves
to escape the interaction. In the case when \(A_H,A_P\neq0\) limit
cycles can emerge.

\hypertarget{space}{%
\subsection{Space}\label{space}}

In this case mean traits become functions of spatial variables (eg.,
\(\bar z_H(x),\bar z_P(x)\) where \(x=(x_1,x_2)\in\mathbb R^2\)). As a
first step, we assume all model parameters are spatially homogeneous. If
we assume interactions occur locally (individuals interact only when
they \enquote{collide}) and abundance densities are spatially
homogeneous, then we can obtain the following continuous space model:

\begin{subequations}\label{deterministic}
  \begin{equation}
    \frac{\partial\bar z_H}{\partial t}=G_HA_H(\theta_H-\bar z_H)-G_HB_H(\bar z_P-\bar z_H)+\frac{D_H}{2}\Delta\bar z_H,
  \end{equation}
  \begin{equation}
    \frac{\partial\bar z_P}{\partial t}=G_PA_P(\theta_P-\bar z_P)+G_PB_P(\bar z_H-\bar z_P)+\frac{D_P}{2}\Delta\bar z_P,
  \end{equation}
\end{subequations}

where \(D_H,D_P>0\) are dispersal rates for each species and
\(\Delta=\frac{\partial^2}{\partial x_1^2}+\frac{\partial^2}{\partial x_2^2}\)
is the diffusion operator representing spatial movement.

\hypertarget{random-genetic-drift}{%
\subsection{Random Genetic Drift}\label{random-genetic-drift}}

To incorporate the effects of random genetic drift, we should be able to
justify the following SPDE model:

\begin{subequations}\label{spde}
  \begin{equation}
    \frac{\partial\bar Z_H}{\partial t}=G_HA_H(\theta_H-\bar Z_H)-G_HB_H(\bar Z_P-\bar Z_H)+\frac{D_H}{2}\Delta\bar Z_H+\sqrt\frac{G_H}{N_H}\dot W_H,
  \end{equation}
  \begin{equation}
    \frac{\partial\bar Z_P}{\partial t}=G_PA_P(\theta_P-\bar Z_P)+G_PB_P(\bar Z_H-\bar Z_P)+\frac{D_P}{2}\Delta\bar Z_P+\sqrt\frac{G_P}{N_P}\dot W_P,
  \end{equation}
\end{subequations}

where \(N_P,N_H\) are the local abundance densities (here assumed to be
constant in space and time), \(\dot W_H,\dot W_P\) are space-time white
noise processes and we use capitol \(\bar Z_H,\bar Z_P\) to emphasize
that these are now random quantities. This SPDE model may possess some
major benefits over more traditional continuous-space models of
population genetics. In particular, it is known to have solutions for
all spatial dimensions. The effect of abiotic stabilizing selection may
also imply function-valued solutions.

\begin{table}[htbp]
\caption{Summary of notation.}
\label{Table:Parameters}
\centering
\begin{tabular}{lll}\hline
Variable/Parameter          & Description                                  & Range                                                     \\ \hline \\
$x=(x_1,x_2)$               & Spatial coordinates                          & $x\in\mathbb R^2$                                         \\ \\
$z_H,z_P$                   & Individual trait values                      & $z_H,z_P\in\mathbb R$                                     \\ \\
$w_H(z_H,z_P),w_P(z_P,z_H)$ & Individual fitness                           & $w_H,w_P>0$                                               \\ \\
$A_H,A_P$                   & Strengths of abiotic stabilizing selection   & $A_H,A_P\geq0$                                            \\ \\
$\theta_H(x),\theta_P(x)$   & Abiotic trait optima at location $x$         & $\theta_H(x),\theta_P(x)\in\mathbb R$                     \\ \\
$B_H,B_P$                   & Strengths of biotic selection                & $B_H,B_P\geq0$                                            \\ \\
$G_H,G_P$                   & Additive genetic variances                   & $G_H,G_P\geq0$                                            \\ \\
$D_H,D_P$                   & Dispersal coefficients                       & $D_H,D_P\geq0$                                            \\ \\
$N_H,N_P$                   & Local effective population sizes             & $N_H,N_P\geq0$                                            \\ \\
$\dot W_H,\dot W_P$         & Independent space-time white noise processes & $\dot W_H(x),\dot W_P(x)$ random variables in $\mathbb R$ \\ \\
$\bar Z_H(x),\bar Z_P(x)$   & Local mean traits at $x$                     & $\bar Z_H(x),\bar Z_P(x)$ random variables in $\mathbb R$ \\ \\
$\bar z_H(x),\bar z_P(x)$   & Expected local mean traits at $x$            & $\bar z_H(x),\bar z_P(x)\in\mathbb R$                     \\ \\ \hline
\end{tabular}
\bigskip{}
\end{table}

\hypertarget{extensions}{%
\subsection{Extensions}\label{extensions}}

\hypertarget{heterogeneous-aboitic-optima}{%
\subsubsection{Heterogeneous aboitic
optima}\label{heterogeneous-aboitic-optima}}

\begin{itemize}
\tightlist
\item
  linear clines, eg:
  \(\theta_H(x)=\alpha_H+\beta_{H,1}x_1+\beta_{H,2}x_2\)
\item
  quadratic clines, eg:
  \(\theta_H(x)=\alpha_H+\beta_{H,1}x_1+\beta_{H,2}x_2+\gamma_{H,11}x_1^2+\gamma_{H,22}x_2^2+\gamma_{H,12}x_1x_2\)
\item
  stochastic fields, eg: \(\theta_H(x)\) follows a fractional Brownian
  surface, but is fixed in time
\end{itemize}

\hypertarget{methods}{%
\section{Methods}\label{methods}}

To study the continuous-space signature of coevolution, we can start
with an analytical approach. However, if the corresponding calculations
prove to be analytically intractable or time costly, we can employ
simulation-based methods to describe the shape of the autocorrelations
and cross-correlation in relevant regions of parameters space. The
analytical approach would begin by identifying stable attractors of the
deterministic system using classical PDE theory. Following this, we may
possibly apply theory of SPDE to describe stationary distributions
associated with each attractor. This may lend analytical solutions for
the expectation and variance of intraspecific autocorrelations and the
interspecific cross-correlation, which are defined in Appendix
\ref{corr-defs} below.

\hypertarget{nondimensional-analysis}{%
\subsection{Nondimensional Analysis}\label{nondimensional-analysis}}

Nondimensional analysis is an approach to simplify mathematical models
by transforming them into equations without units. In particular, traits
\(\bar Z_H,\bar Z_P\) are in phenotypic units, additive genetic
variances \(G_H,G_P\) are in squared phenotypic units, spatial
coordinates \(x=(x_1,x_2)\) are in units of distance and dispersal
coefficients \(D_H,D_P\) are in units of distance squared. By removing
these units, we can pinpoint characteristic scales emerging from the
interactions of different processes. From Murray 2003:

\begin{displayquote}
  One of the most useful aspects of a nondimensional analysis, with resulting nondimensional groupings of the parameters, is that it is possible to assess how different physical effects, quantified by the parameters in the nondimensionalisation, trade off against one another.
\end{displayquote}

Hence, we can use nondimensional analysis to assess how selection,
dispersal and drift interact to determine characteristic spatial scales.
The difficulty here is that there's two species with different dispersal
coefficients. This leads to two distinct characteristic spatial scales
at which patterns emerge. Perhaps we can focus on patterns of
matching/mis-matching by tracking \(Y(x):=\bar Z_H(x)-\bar Z_P(x)\)? The
trouble with this is that the species are evolving to different
strengths of abiotic selection. Hence, it's not clear what rescaling
would nondimensionalize \(Y\).

\hypertarget{discussion}{%
\section{Discussion}\label{discussion}}

\hypertarget{a-note-on-biological-diversity-vs-statistical-uncertainty}{%
\subsection{A Note on Biological Diversity vs Statistical
Uncertainty}\label{a-note-on-biological-diversity-vs-statistical-uncertainty}}

An important distinction to make when analyzing spatial patterns of
coevolving species is the difference between statistical uncertainty and
biological variation across space. In particular, we can ignore the
stochastic effects of random genetic drift and focus on patterns
produced by deterministic processes through studying the PDE
(\ref{deterministic}). In this case, we might study quantities such as
the global mean trait values (ie., mean trait values averaged across
space) and the spatial variance of mean trait values. These quantities
are purely functions of existing biological diversity and do not take
into account uncertainties inherit in the process of evolution (eg.,
random genetic drift). Once drift is accounted for, the global mean and
spatial variance of local mean trait values themselves become random
variables.

To understand why spatial expectations and variances become random
variables in the presence of drift, we can think of each possible
instance of the white-noise processes \(\dot W_H,\dot W_P\). In
probability theory these instances are called sample paths and are
typically denoted by the variable \(\omega\). The set of all \(\omega\)
is denoted \(\Omega\) and is referred to as the probability space.
Hence, we would refer to \(\dot W_H(\omega),\dot W_P(\omega)\) as
particular sample paths of the processes \(\dot W_H,\dot W_P\). This
implies that a solution to the stochastic system (\ref{spde}) is really
a set of solutions indexed by \(\omega\in\Omega\). For each \(\omega\)
we can compute spatial averages, variances and covariance of mean
traits. Hence, the probabilistic expectation and variance of these
spatial statistics are found by integrating across \(\Omega\).

This heavily theoretical discussion is important for understanding the
difference between spatial autocorrelation and the covariance function
of a spatial stochastic process. In particular, the spatial
autocorrelation we are interested in (denoted \(\rho_{ff}(r)\) for a
spatial function \(f\)) corresponds to deterministic patterns. Hence,
when accounting for random genetic drift, \(\rho_{ff}\) will be a random
function. To compute the expectation, variance and covariance of this
function, we can use properties of the associated spatial stochastic
process (the expectation and covariance function, in particular).

Finally, for the sake of completion it is worthwhile to note that
sampling error adds a second layer of stochasticity.

\newpage

\begin{center}
  \Large Appendices
\end{center}

\appendix

\hypertarget{spatial-moments}{%
\section{Spatial Moments}\label{spatial-moments}}

Suppose \(f(x)\) is a function of a two-dimensional spatial parameter
\(x=(x_1,x_2)\). For simplicity, let's consider a rectangular subset
\(\Gamma\) of \(\mathbb R^2\) with periodic boundaries (ie., the surface
of a torus). Then we can compute the spatial average of \(f(x)\) via

\begin{equation}
  \bar f = \frac{1}{|\Gamma|}\int_\Gamma f(x)dx,
\end{equation} where \(|\Gamma|\) is the size of the subset \(\Gamma\).
The spatial variance of \(f\) is then

\begin{equation}
  \mathrm{Var}(f) = \frac{1}{|\Gamma|}\int_\Gamma (f(x)-\bar f)^2dx,
\end{equation}

\hypertarget{spatial-autocorrelation-and-cross-correlation}{%
\subsection{\texorpdfstring{Spatial Autocorrelation and
Cross-correlation
\label{corr-defs}}{Spatial Autocorrelation and Cross-correlation }}\label{spatial-autocorrelation-and-cross-correlation}}

The cross-correlation of two functions \(f,g\) is defined by

\begin{equation}
  \rho_{fg}(x)=\int_\Gamma f(y)g(y-x)dy.
\end{equation}

We can reduce this quantity to a function of distance
\(r=\sqrt{x_1^2+x_2^2}\) using the average

\begin{equation}
  \bar\rho_{fg}(r)=\frac{1}{2\pi r}\int_0^{2\pi}\rho_{fg}((r\cos\theta,r\sin\theta))d\theta.
\end{equation}

The autocorrelation of \(f\) is then \(\rho_{ff}\).

\hypertarget{cross-covariance}{%
\subsection{Cross-covariance}\label{cross-covariance}}

Following the above definitions of spatial moments and
cross-correlation, we compute the cross-covariance of two spatial
functions \(f\) and \(g\) (in 2D space) via

\begin{equation}
  K_{fg}(x)=\sqrt{\mathrm{Var}(f)\mathrm{Var}(g)}\rho_{f,g}(x).
\end{equation}

Hence, inspired by the definition of \(\bar\rho_{fg}\), we can define
the cross covariance between spatial functions \(f\) and \(g\) at two
locations separated by distance \(r\) via

\begin{equation}
  \bar K_{fg}(r)=\frac{1}{2\pi r}\sqrt{\mathrm{Var}(f)\mathrm{Var}(g)}\int_0^{2\pi}K_{fg}((r\cos\theta,r\sin\theta))d\theta.
\end{equation}

\hypertarget{a-stable-equilibrium-of-pde}{%
\section{\texorpdfstring{A Stable Equilibrium of PDE
(\ref{deterministic})}{A Stable Equilibrium of PDE ()}}\label{a-stable-equilibrium-of-pde}}

To find an equilibrium of the deterministic PDE (\ref{deterministic}) we
set the time derivatives of \(\bar z_H\) and \(\bar z_P\) equal to zero.
This leads to following systems of ODE's

\begin{subequations}
  \begin{equation}
    0=G_HA_H(\theta_H-\bar z_H)-G_HB_H(\bar z_P-\bar z_H)+\frac{D_H}{2}\Delta\bar z_H,
  \end{equation}
  \begin{equation}
    0=G_PA_P(\theta_P-\bar z_P)+G_PB_P(\bar z_H-\bar z_P)+\frac{D_P}{2}\Delta\bar z_P.
  \end{equation}
\end{subequations}

Solving this system for the mean traits \(\bar z_H(x_1,x_2)\) and
\(\bar z_P(x_1,x_2)\) as spatial functions then returns equilibrium
solutions to PDE (\ref{deterministic}). However, we can start simple by
considering spatially homogeneous solutions. In these cases the spatial
derivatives will return zero (in particular,
\(\Delta\bar z_H=\Delta\bar z_P=0\)). Finding these solutions amounts to
solve the following system of algebraic equations

\begin{subequations}
  \begin{equation}
    0=G_HA_H(\theta_H-\bar z_H)-G_HB_H(\bar z_P-\bar z_H),
  \end{equation}
  \begin{equation}
    0=G_PA_P(\theta_P-\bar z_P)+G_PB_P(\bar z_H-\bar z_P).
  \end{equation}
\end{subequations}

This linear system is uniquely solved by

\begin{subequations}\label{non-spatial-eq}
  \begin{equation}
    \bar z_H(x_1,x_2)\equiv\frac{(A_P+B_P)A_H\theta_H-B_HA_P\theta_P}{(A_P+B_P)A_H-B_HA_P},
  \end{equation}
  \begin{equation}
    \bar z_P(x_1,x_2)\equiv\frac{(A_H-B_H)A_P\theta_P+B_PA_H\theta_H}{(A_H-B_H)A_P+B_PA_H},
  \end{equation}
\end{subequations}

where the symbol \(\equiv\) is used in place of \(=\) to emphasize these
functions are constant across space.

To understand whether this equilibrium is stable we perform stability
analysis on the non-spatial model. Notice that system
(\ref{non-spatial}) can be rewritten in matrix notation as
\(\vec{\bar z}=M\vec{\bar z}+\vec b\) where

\begin{subequations}
  \begin{equation}
    \vec{\bar z}=\left(\begin{matrix}
      \bar z_H \\ \bar z_P
    \end{matrix}\right), \ 
    \vec b =\left(\begin{matrix}
      G_HA_H\theta_H \\ G_PA_P\theta_P
    \end{matrix}\right),
  \end{equation}
  \begin{equation}
    M=\left(\begin{matrix}
      G_H(B_H-A_H) & -G_HB_H \\
      G_PB_P & -G_P(A_P+B_P)
    \end{matrix}\right).
  \end{equation}
\end{subequations}

It turns that if the real parts of the eigenvalues of \(M\) are negative
then equilibrium (\ref{non-spatial-eq}) is stable. This condition holds
in general when \(B_H<A_H+(A_P+B_P)G_P/G_H\) and \(B_H<A_H(1+B_P/A_P)\).
A sufficient condition with clear biological intuition is \(B_H<A_H\),
which just means the strength of biotic selection on the host is less
than the strength of abiotic stabilizing selection. In this case abiotic
stabilizing selection prevents the host from escaping via evolution.

\hypertarget{stationary-solutions-of-spde}{%
\section{\texorpdfstring{Stationary Solutions of SPDE
(\ref{spde})}{Stationary Solutions of SPDE ()}}\label{stationary-solutions-of-spde}}

To focus on stationary solutions of the stochastic system (\ref{spde})
we set the time derivatives to zero to obtain the following system

\begin{subequations}\label{stationary-spde}
  \begin{equation}
    G_HA_H(\theta_H-\bar Z_H)-G_HB_H(\bar Z_P-\bar Z_H)+\frac{D_H}{2}\Delta\bar Z_H=\sqrt\frac{G_H}{N_H}\dot W_H,
  \end{equation}
  \begin{equation}
    G_PA_P(\theta_P-\bar Z_P)+G_PB_P(\bar Z_H-\bar Z_P)+\frac{D_P}{2}\Delta\bar Z_P=\sqrt\frac{G_P}{N_P}\dot W_P.
  \end{equation}
\end{subequations}

If we further restrict our focus to the single species case we recover a
SPDE of the form

\begin{equation}\label{simple-form}
  b^2(a-u)+\Delta u=\sigma\dot W.
\end{equation}

In the case of two spatial dimensions, it is known that equation
(\ref{simple-form}) is satisfied by a Gaussian field with a Whittle
covariance function (Whittle 1963; Sigrist, Künsch, and Stahel 2014).
Denoting \(K_1\) the modified Bessel function of the second kind, order
1, the Whittle covariance function is given by
\(C(r)=\frac{\sigma^2r}{2b}K_1(br)\). Hence, we can postulate that
solutions of the stationary system (\ref{stationary-spde}) will have
Whittle cross-covariance matrix-valued functions.

\hypertarget{the-autocovariances-and-cross-covariance-of-bar-z_h-and-bar-z_p}{%
\section{\texorpdfstring{The autocovariances and cross-covariance of
\(\bar Z_H\) and
\(\bar Z_P\)}{The autocovariances and cross-covariance of \textbackslash bar Z\_H and \textbackslash bar Z\_P}}\label{the-autocovariances-and-cross-covariance-of-bar-z_h-and-bar-z_p}}

Set \(\bar z_H(x)=\mathbb E[\bar Z_H(x)]\) and
\(\bar z_P(x)=\mathbb E[\bar Z_P(x)]\). Since the system
(\ref{stationary-spde}) is linear, we can expect \((\bar z_H,\bar z_P)\)
to solve the associated deterministic system. Then, it seems reasonable
to expect the cross-covariance of \(\bar Z_H\) and \(\bar Z_P\) across
space to satisfy

\begin{equation}
  K_{\bar Z_H\bar Z_P}(x)=K_{\bar z_H\bar z_P}(x)+\mathcal{A}_{HP}(x)+\mathcal B_{HP}(x),
\end{equation}

where \(\mathcal{A}_{HP}(x)\) is the cross-covariance of \(\bar Z_H\)
and \(\bar Z_P\) generated by abiotic stabilizing selection and
\(\mathcal B_{HP}(x)\) is the cross-covariance generated by
coevolutionary selection.

\begin{itemize}
\tightlist
\item
  \textbf{Note:} In the case of the stable equilibrium considered above,
  \(\bar z_H\) and \(\bar z_P\) are spatially homogeneous which implies
  \(K_{\bar z_H\bar z_P}(x)\equiv\sqrt{\mathrm{Var}(\bar z_H)\mathrm{Var}(\bar z_P)}=0\)
  since \(\mathrm{Var}(\bar z_H)=\mathrm{Var}(\bar z_P)=0\).
\end{itemize}

Similarly, we might expect the autocovariances of \(\bar Z_H\) and
\(\bar Z_P\) to satisfy

\begin{equation}
  K_{\bar Z_H\bar Z_H}(x)=K_{\bar z_H\bar z_H}(x)+\mathcal{A}_H(x)+\mathcal B_H(x),
\end{equation} \begin{equation}
  K_{\bar Z_P\bar Z_P}(x)=K_{\bar z_P\bar z_P}(x)+\mathcal{A}_P(x)+\mathcal B_P(x),
\end{equation}

where \(\mathcal{A}_X(x)\) is the autocovariance of species \(X\)
generated by abiotic stabilizing selection and \(\mathcal{B}_X(x)\) is
the autocovariance generated by unilateral biotic selection on species
\(X\). In general, the signature of coevolution is likely to be found in
the six covariance functions:
\(K_{\bar z_H\bar z_P}(x),K_{\bar z_H\bar z_H}(x),K_{\bar z_P\bar z_P}(x),\mathcal B_{HP}(x),\mathcal B_H(x)\)
and \(\mathcal B_P(x)\).

\hypertarget{nondimensionalization}{%
\section{Nondimensionalization}\label{nondimensionalization}}

Our approach to nondimensionalize system (\ref{spde}) is borrowed from
Polechova \& Barton (2015). In particular, these authors
nondimensionalize the SPDE

\[\frac{\partial\bar Z}{\partial t}=G\frac{\partial\bar r}{\partial\bar Z}+\frac{D}{2}\Delta\bar z+\sqrt{\frac{G}{N}}\dot W,\]

where \(\bar r\) is the population growth rate and its derivative with
respect to \(\bar Z\) represents selection.

\hypertarget{more}{%
\section{More?}\label{more}}

\begin{itemize}
\tightlist
\item
  Explore other patterns of spatial covariance using different models of
  spatial movement.

  \begin{itemize}
  \tightlist
  \item
    This leads to other random surfaces such as fractional Brownian
    surfaces which are characterized by different families of covariance
    functions.
  \item
    Hence, it may be possible to learn models of spatial movement based
    on patterns of spatial covariance.
  \end{itemize}
\item
  What happens when interacting species have different movement models?
\end{itemize}

\hypertarget{application}{%
\section{Application}\label{application}}

\begin{itemize}
\tightlist
\item
  How tough would it be to revisit toju and pauw data-sets to infer
  coevolution again?
\end{itemize}

\newpage

\hypertarget{applying-hu2013multivariate-to-compute-covariance-functions}{%
\section{Applying (Hu et al. 2013) to Compute Covariance
Functions}\label{applying-hu2013multivariate-to-compute-covariance-functions}}

In the technical note of (Hu et al. 2013), a method to compute
covariance functions of multivariate Gaussian random fields from systems
of SPDE was outlined. The approach begins with a stationary system such
as (\ref{stationary-spde}) and applies a Fourier transform
\(\mathcal F\) to convert derivatives into algebraic expressions. The
Fourier transform acts to switch perspective from the two-dimensional
spatial variable \(x=(x_1,x_2)\) to a two-dimensional frequency variable
\(s=(s_1,s_2)\), where \(s_1,s_2\) are complex numbers. We can provide a
spectral characterization of the system by considering the behavior of a
system across a spectrum of frequencies. Once solving for the quantity
of interest in the spectral representation we can use the inverse
Fourier transform \(\mathcal F^{-1}\) to obtain the associated quantity
as a function of space.

To apply this method to our model, we start by using a change of
variables to obtain an equivalent system where each spatial variable has
zero mean. In particular, we set \(\bar\zeta_H=\bar Z_H-\bar z_H\) and
\(\bar\zeta_P=\bar Z_P-\bar z_p\) where \((\bar z_H,\bar z_P)\) is the
spatially homogeneous equilibrium of the deterministic system reported
in equation (\ref{non-spatial-eq}). Under this change of variables, an
Itô formula can be applied (need to fact-check) to show the stationary
system (\ref{stationary-spde}) becomes

\begin{subequations}
  \begin{equation}
    \left(\frac{D_H}{2}\Delta-G_H(A_H-B_H)\right)\bar\zeta_H-G_HB_H\bar\zeta_P=\sqrt\frac{G_H}{N_H}\dot W_H,
  \end{equation}
  \begin{equation}
    G_PB_P\bar \zeta_H+\left(\frac{D_P}{2}\Delta-G_P(A_P+B_P)\right)\bar\zeta_P=\sqrt\frac{G_P}{N_P}\dot W_P.
  \end{equation}
\end{subequations}

To consolidate notation, we set

\begin{subequations}
  \begin{equation}
    \pmb{\mathscr{L}} = \left(\begin{matrix}
      G_H(A_H-B_H)-\frac{D_H}{2}\Delta & G_HB_H \\ & \\
      -G_PB_P & G_P(A_P+B_P)-\frac{D_P}{2}\Delta
    \end{matrix}\right),
  \end{equation}
  \begin{equation}
    \bar{\pmb{\zeta}} = \left(\begin{matrix}
      \bar\zeta_H \\ \\ \bar\zeta_P
    \end{matrix}\right), \ 
    \pmb{V} = \left(\begin{matrix}
      -\sqrt\frac{G_H}{N_H}\dot W_H \\ \\ 
      -\sqrt\frac{G_P}{N_P}\dot W_P
    \end{matrix}\right).
  \end{equation}
\end{subequations}

Then, the stationary system can equally be written as

\begin{equation}
  \pmb{\mathscr{L}}\bar{\pmb\zeta}=\pmb V.
\end{equation}

Setting
\(\hat\zeta_H=\mathcal{F}[\bar\zeta_H], \ \hat\zeta_P=\mathcal{F}[\bar\zeta_P], \hat V_H=\mathcal{F}[-\sqrt{\frac{G_H}{N_H}}\dot W_H], \ \hat V_P=\mathcal{F}[-\sqrt{\frac{G_P}{N_P}}\dot W_P]\),
we can Fourier transform the whole dang thing to get

\begin{equation}
  \pmb{\mathcal{H}}\hat{\pmb\zeta}=\hat{\pmb V},
\end{equation}

where

\begin{subequations}
  \begin{equation}
    \pmb{\mathscr H}=\left(\begin{matrix}
      G_H(A_H-B_H)+\frac{D_H}{2}\|s\|^2 & G_HB_H \\ & \\
      -G_PB_P & G_P(A_P+B_P)+\frac{D_P}{2}\|s\|^2
    \end{matrix}\right),
  \end{equation}
  \begin{equation}
    \hat{\pmb\zeta}=\left(\begin{matrix}
      \hat\zeta_H \\ \\ \hat\zeta_P
    \end{matrix}\right), \\
    \hat{\pmb V}=\left(\begin{matrix}
      \hat V_H \\ \\ \hat V_P
    \end{matrix}\right), \\
  \end{equation}
\end{subequations}

The power spectrum corresponding to \(\bar{\pmb\zeta}\) is defined as
\(\pmb S_{\bar{\pmb\zeta}}=\mathbb E\left[\hat{\pmb\zeta}\hat{\pmb\zeta}^\top\right]\),
where \(\top\) denotes matrix transposition. In particular, this
provides

\begin{equation}
  \pmb S_{\bar{\pmb\zeta}}=\left(\begin{matrix}
    S_{\bar\zeta_{HH}} & S_{\bar\zeta_{HP}} \\ & \\
    S_{\bar\zeta_{PH}} & S_{\bar\zeta_{PP}}
  \end{matrix}\right)=\left(\begin{matrix}
    \mathbb E[\hat\zeta_H\hat\zeta_H] & \mathbb E[\hat\zeta_H\hat\zeta_P] \\ & \\
    \mathbb E[\hat\zeta_H\hat\zeta_P] & \mathbb E[\hat\zeta_P\hat\zeta_P]
  \end{matrix}\right)=\pmb{\mathscr H}^{-1}\pmb S_{\pmb V}\pmb{\mathscr H}^{-H},
\end{equation}

with \(-H\) denoting the inverse of the Hermitian of a matrix and
\(\pmb S_{\pmb V}=\mathbb E\left[\hat{\pmb V}\hat{\pmb V}^H\right]\)
denoting the power spectrum of the noise process \(\pmb V\). Using
properties of Fourier transforms, we compute

\begin{equation}
  \hat{\pmb V}=\left(\begin{matrix}
    (2\pi)^{-1}\sqrt{G_H/N_H} \\ \\
    (2\pi)^{-1}\sqrt{G_P/N_P}
  \end{matrix}\right).
\end{equation}

Since we assume the white noise processes \(\dot W_H\) and \(\dot W_P\)
are independent we have

\begin{equation}
  \pmb S_{\pmb V}=\left(\begin{matrix}
    S_{V_H} & 0 \\ & \\
    0 & S_{V_P}
  \end{matrix}\right)=\left(\begin{matrix}
    (2\pi)^{-2}G_H/N_H & 0 \\ & \\
    0 & (2\pi)^{-2}G_P/N_P
  \end{matrix}\right).
\end{equation}

Putting it all together, we compute \(\pmb S_{\pmb{\bar\zeta}}\) from

\begin{subequations}
  \begin{equation}
    S_{\bar\zeta_{HH}}=\frac{S_{V_H}|\mathscr H_{22}^2|+S_{V_P}|\mathscr H_{12}^2|}{|(\mathscr H_{11}\mathscr H_{22}-\mathscr H_{12}\mathscr H_{21})^2|},
  \end{equation}
  \begin{equation}
    S_{\bar\zeta_{HP}}=-\frac{\mathscr H_{22}S_{V_H}|\mathscr H_{21}^2|\mathscr H_{11}+\mathscr H_{12}S_{V_P}|\mathscr H_{11}^2|\mathscr H_{21}}{|(\mathscr H_{11}\mathscr H_{22}-\mathscr H_{12}\mathscr H_{21})^2|\mathscr H_{21}\mathscr H_{11}},
  \end{equation}
  \begin{equation}
    S_{\bar\zeta_{PH}}=-\frac{\mathscr H_{21}S_{V_H}|\mathscr H_{22}^2|\mathscr H_{12}+\mathscr H_{11}S_{V_P}|\mathscr H_{12}^2|\mathscr H_{22}}{|(\mathscr H_{11}\mathscr H_{22}-\mathscr H_{12}\mathscr H_{21})^2|\mathscr H_{22}\mathscr H_{12}},
  \end{equation}
  \begin{equation}
    S_{\bar\zeta_{PP}}=\frac{S_{V_H}|\mathscr H_{21}^2|+S_{V_P}|\mathscr H_{11}^2|}{|(\mathscr H_{11}\mathscr H_{22}-\mathscr H_{12}\mathscr H_{21})^2|}.
  \end{equation}
\end{subequations}

\hypertarget{multivariate-matern-covariance-function}{%
\subsection{Multivariate Matern covariance
function}\label{multivariate-matern-covariance-function}}

The Matern covariance matrix function (from Gneiting et al 2010)

\[C(\pmb x)=\left(\begin{matrix}
  \sigma_1^2M(x|\nu_1,a_2) & \rho_{12}\sigma_1\sigma_2M(x|\nu_{12},a_{12}) \\ & \\
  \rho_{12}\sigma_1\sigma_2M(x|\nu_{12},a_{12}) & \sigma_2^2M(x|\nu_2,a_2)
\end{matrix}\right)\]

\[C(\pmb x)=\left(\begin{matrix}
  a_H\sigma_H^2\|\pmb x\|K_1(a_H\|\pmb x\|) & a_{HP}\rho_{HP}\sigma_H\sigma_P\|\pmb x\|K_1(a_{HP}\|\pmb x\|) \\ & \\
  a_{PH}\rho_{PH}\sigma_H\sigma_P\|\pmb x\|K_1(a_{PH}\|\pmb x\|) & a_P\sigma_P^2\|\pmb x\|K_1(a_P\|\pmb x\|)
\end{matrix}\right)\]

has power spectrum matrix

\newpage

\hypertarget{references}{%
\section*{References}\label{references}}
\addcontentsline{toc}{section}{References}

\hypertarget{refs}{}
\leavevmode\hypertarget{ref-hu2013multivariate}{}%
Hu, Xiangping, Daniel Simpson, Finn Lindgren, and Håvard Rue. 2013.
``Multivariate Gaussian Random Fields Using Systems of Stochastic
Partial Differential Equations.'' \emph{arXiv Preprint arXiv:1307.1379}.

\leavevmode\hypertarget{ref-Sigrist2014}{}%
Sigrist, Fabio, Hans R. Künsch, and Werner A. Stahel. 2014. ``Stochastic
Partial Differential Equation Based Modelling of Large Space-Time Data
Sets.'' \emph{Journal of the Royal Statistical Society: Series B
(Statistical Methodology)} 77 (1): 3--33.
\url{https://doi.org/10.1111/rssb.12061}.

\leavevmode\hypertarget{ref-whittle1963stochastic}{}%
Whittle, Peter. 1963. ``Stochastic-Processes in Several Dimensions.''
\emph{Bulletin of the International Statistical Institute} 40 (2):
974--94.

\bibliographystyle{unsrt}
\bibliography{references.bib}


\end{document}
