\documentclass{article}

\usepackage{arxiv}

\usepackage[utf8]{inputenc} % allow utf-8 input
\usepackage[T1]{fontenc}    % use 8-bit T1 fonts
\usepackage{lmodern}        % https://github.com/rstudio/rticles/issues/343
\usepackage{hyperref}       % hyperlinks
\usepackage{url}            % simple URL typesetting
\usepackage{booktabs}       % professional-quality tables
\usepackage{amsfonts}       % blackboard math symbols
\usepackage{nicefrac}       % compact symbols for 1/2, etc.
\usepackage{microtype}      % microtypography
\usepackage{lipsum}
\usepackage{graphicx}

\title{The Phenotypic Signature of Host-Parasite Coevolution in Continuous
Space}

\author{
    Bob Week
   \\
    Integrative Biology \\
    Michigan State University \\
  East Lansing, MI 48824 \\
  \texttt{\href{mailto:weekrobe@msu.edu}{\nolinkurl{weekrobe@msu.edu}}} \\
   \And
    Gideon Bradburd
   \\
    Integrative Biology \\
    Michigan State University \\
  East Lansing, MI 48824 \\
  \texttt{\href{mailto:bradburd@msu.edu}{\nolinkurl{bradburd@msu.edu}}} \\
  }


% Pandoc citation processing

\usepackage{amsmath}
\usepackage{mathrsfs}
\usepackage{csquotes}
\usepackage{textcomp}


\begin{document}
\maketitle

\def\tightlist{}


\begin{abstract}
Here we identify the phenotypic signature of host-parasite coevolution
in continuous space.
\end{abstract}

\keywords{
    blah
   \and
    blee
   \and
    bloo
   \and
    these are optional and can be removed
  }

\hypertarget{introduction}{%
\section{Introduction}\label{introduction}}

Space is hypothesized to play a fundamental role in the coevolutionary
process. In particular, the combined effects of local co-adaptation and
spatial movement are thought to produce complex spatial patterns of
matched and mismatched traits. However, theoretical studies
investigating the spatial signal of coevolution have for the most part
restricted themselves to understanding patterns of spatial correlation
using spatially implicit models. Hence, the fine-grained details that
can be produced in spatially explicit settings have yet to be
understood. Here we close this gap in the context of host-parasite
coevolution using continuous-space models inspired by classical
quantitative genetic theory. In particular, we study the
autocorrelations of mean traits within species and the cross-correlation
of mean traits between species when trait dynamics are driven by (1)
random genetic drift, gene-flow and abiotic stabilizing selection, (2)
drift, gene-flow, abiotic stabilizing selection and unilateral
adaptation of the host species, (3) drift, gene-flow, stabilizing
selection and unilateral adaptation of the parasite species and (4)
drift, gene-flow, stabilizing selection and host-parasite coevolution.
In each case we assuming abiotic stabilizing selection is spatially
homogeneous. By comparing patterns of spatial autocorrelation and
cross-correlation in these four scenarios we are able to elucidate the
signature of coevolution.

\hypertarget{questions}{%
\section{Questions}\label{questions}}

\begin{itemize}
\tightlist
\item
  Are we aiming to describe the spatial signature of coevolution? Or are
  we aiming to explain spatial patterns of traits and local adaptation
  when species may be coevolving?

  \begin{itemize}
  \tightlist
  \item
    The first question takes a more reductionist view by focusing on
    coevolution.
  \item
    The second question takes a more holistic view by embracing the
    complexity of natural systems.
  \end{itemize}
\item
  What are the characteristic spatial scales of local adaptation in each
  species?

  \begin{itemize}
  \tightlist
  \item
    How do these depend on rates/distances of dispersal and strengths of
    selection?
  \end{itemize}
\item
  What is the characteristic spatial scale at which coevolution becomes
  visible?

  \begin{itemize}
  \tightlist
  \item
    How does this depend on dispersal distances, strengths of selection,
    local effective sizes and local additive genetic variation?
  \end{itemize}
\item
  What are the characteristic scales of spatial autocorrelation and
  cross-correlation?

  \begin{itemize}
  \tightlist
  \item
    Does this answer both questions above?
  \end{itemize}
\item
  How do these answers change when selection, local effective sizes or
  local additive genetic variances are spatially heterogeneous (but
  temporally fixed)?

  \begin{itemize}
  \tightlist
  \item
    What if these parameters follow deterministic clines? Linear or
    quadratic?
  \item
    What if they follow stochastic patterns such as Gaussian fields?

    \begin{itemize}
    \tightlist
    \item
      In particular, consider the characteristic scale of environmental
      heterogeneity.
    \end{itemize}
  \item
    In these scenarios, how does autocorrelation and cross-correlation
    change across space? What is the characteristic scale at which their
    characteristic scales change?
  \end{itemize}
\item
  Considering answers to all of the above, what is the best sampling
  scheme for detecting coevolution?

  \begin{itemize}
  \tightlist
  \item
    More intensive sampling at a few locations?
  \item
    Less intensive sampling at many locations?
  \end{itemize}
\item
  Future directions:

  \begin{itemize}
  \tightlist
  \item
    Does the interplay between coevolution and abundance dynamics lead
    to patchy distributions?
  \item
    When does local adaptation occur in host and/or parasite?

    \begin{itemize}
    \tightlist
    \item
      Are there winners/losers? (ref scott perspective piece)
    \item
      When do host or parasite remain generalized?
    \item
      Does this agree with past theory?

      \begin{itemize}
      \tightlist
      \item
        I imagine the stochastic dynamics change the picture
      \end{itemize}
    \end{itemize}
  \end{itemize}
\end{itemize}

\hypertarget{the-model}{%
\section{The Model}\label{the-model}}

\hypertarget{fitness}{%
\subsection{Fitness}\label{fitness}}

The host is denoted \(H\) and the parasite \(P\). Individual trait
values of hosts and parasites are respectively denoted by
\(z_H,z_P\in\mathbb R\). Assuming a trait-matching model, where host
fitness is minimized and parasite fitness is maximized when \(z_H=z_P\),
we have the following two fitness functions:

\begin{subequations}\label{fit}
  \begin{equation}
    w_H(z_H,z_P)\propto \exp\left(-\frac{A_H}{2}(\theta_H-z_H)^2+\frac{B_H}{2}(z_P-z_H)^2\right),
  \end{equation}
  \begin{equation}
    w_P(z_P,z_H)\propto \exp\left(-\frac{A_P}{2}(\theta_P-z_P)^2-\frac{B_P}{2}(z_H-z_P)^2\right),
  \end{equation}
\end{subequations}

where \(A_H,A_P>0\) capture the strengths of abiotic stabilizing
selection, \(\theta_H,\theta_P\in\mathbb{R}\) are the abiotic optimal
phenotypes and \(B_H,B_P>0\) capture the strengths of biotic selection
experienced by each species.

\hypertarget{non-spatial-dynamics}{%
\subsection{Non-spatial Dynamics}\label{non-spatial-dynamics}}

Using a standard approach, we use the above fitness functions to derive
the following non-spatial coevolutionary model:

\begin{subequations}\label{non-spatial}
  \begin{equation}
    \frac{d\bar z_H}{dt}=G_HA_H(\theta_H-\bar z_H)-G_HB_H(\bar z_P-\bar z_H),
  \end{equation}
  \begin{equation}
    \frac{d\bar z_P}{dt}=G_PA_P(\theta_P-\bar z_P)+G_PB_P(\bar z_H-\bar z_P),
  \end{equation}
\end{subequations}

where \(G_H,G_P>0\) denote the additive genetic variances of each
species. In this linear set of equations, when abiotic stabilizing
selection is absent (\(A_H=A_P=0\)), the parasite mean trait
\(\bar z_P\) evolves to match the host mean trait \(\bar z_H\).
Simultaneously \(\bar z_H\) evolves away from \(\bar z_P\). In
particular, when \(\bar z_H>\bar z_P\), the host mean trait will evolve
upwards. Similarly, when \(\bar z_H<\bar z_P\), the host mean trait will
evolve downwards. This and related models have been thoroughly studied
by Sergey Gavrilets and others. For the case when \(A_H=A_P=0\) two
possible outcomes have been demonstrated. First, when \(G_PB_P>G_HB_H\)
so that the parasite evolves faster than the host, the system
asymptotically evolves to the equilibrium \(\bar z_H=\bar z_P\). Second,
when \(G_PB_P<G_HB_H\) so that the host evolves faster than the
parasite, there is no stable equilibrium. This implies the host evolves
to escape the interaction. In the case when \(A_H,A_P\neq0\) limit
cycles can emerge.

\hypertarget{space}{%
\subsection{Space}\label{space}}

In this case mean traits become functions of spatial variables (eg.,
\(\bar z_H(\pmb x),\bar z_P(\pmb x)\) where
\(\pmb x=(x_1,x_2)\in\mathbb R^2\)). As a first step, we assume all
model parameters are spatially homogeneous. If we assume interactions
occur locally (individuals interact only when they \enquote{collide})
and abundance densities are spatially homogeneous, then we can obtain
the following continuous space model:

\begin{subequations}\label{deterministic}
  \begin{equation}
    \frac{\partial\bar z_H}{\partial t}=G_HA_H(\theta_H-\bar z_H)-G_HB_H(\bar z_P-\bar z_H)+\frac{D_H}{2}\Delta\bar z_H,
  \end{equation}
  \begin{equation}
    \frac{\partial\bar z_P}{\partial t}=G_PA_P(\theta_P-\bar z_P)+G_PB_P(\bar z_H-\bar z_P)+\frac{D_P}{2}\Delta\bar z_P,
  \end{equation}
\end{subequations}

where \(D_H,D_P>0\) are dispersal rates for each species and
\(\Delta=\frac{\partial^2}{\partial x_1^2}+\frac{\partial^2}{\partial x_2^2}\)
is the diffusion operator representing spatial movement.

\hypertarget{random-genetic-drift}{%
\subsection{Random Genetic Drift}\label{random-genetic-drift}}

To incorporate the effects of random genetic drift, we should be able to
justify the following SPDE model:

\begin{subequations}\label{spde}
  \begin{equation}
    \frac{\partial\bar Z_H}{\partial t}=G_HA_H(\theta_H-\bar Z_H)-G_HB_H(\bar Z_P-\bar Z_H)+\frac{D_H}{2}\Delta\bar Z_H+\sqrt\frac{G_H}{N_H}\dot W_H,
  \end{equation}
  \begin{equation}
    \frac{\partial\bar Z_P}{\partial t}=G_PA_P(\theta_P-\bar Z_P)+G_PB_P(\bar Z_H-\bar Z_P)+\frac{D_P}{2}\Delta\bar Z_P+\sqrt\frac{G_P}{N_P}\dot W_P,
  \end{equation}
\end{subequations}

where \(N_P,N_H\) are the local abundance densities (here assumed to be
constant in space and time), \(\dot W_H,\dot W_P\) are space-time white
noise processes and we use capitol \(\bar Z_H,\bar Z_P\) to emphasize
that these are now random quantities. This SPDE model may possess some
major benefits over more traditional continuous-space models of
population genetics. In particular, it is known to have solutions for
all spatial dimensions. The effect of abiotic stabilizing selection may
also imply function-valued solutions.

\begin{table}[htbp]
\caption{Summary of notation.}
\label{Table:Parameters}
\centering
\begin{tabular}{lll}\hline
Variable/Parameter                  & Description                                  & Range                                                     \\ \hline \\
$\pmb x=(x_1,x_2)$                  & Spatial coordinates                          & $\pmb x\in\mathbb R^2$                                    \\ \\
$z_H,z_P$                           & Individual trait values                      & $z_H,z_P\in\mathbb R$                                     \\ \\
$w_H(z_H,z_P),w_P(z_P,z_H)$         & Individual fitness                           & $w_H,w_P>0$                                               \\ \\
$A_H,A_P$                           & Strengths of abiotic stabilizing selection   & $A_H,A_P\geq0$                                            \\ \\
$\theta_H(\pmb x),\theta_P(\pmb x)$ & Abiotic trait optima at location $\pmb x$    & $\theta_H(\pmb x),\theta_P(\pmb x)\in\mathbb R$           \\ \\
$B_H,B_P$                           & Strengths of biotic selection                & $B_H,B_P\geq0$                                            \\ \\
$G_H,G_P$                           & Additive genetic variances                   & $G_H,G_P\geq0$                                            \\ \\
$D_H,D_P$                           & Dispersal coefficients                       & $D_H,D_P\geq0$                                            \\ \\
$N_H,N_P$                           & Local effective population sizes             & $N_H,N_P\geq0$                                            \\ \\
$\dot W_H,\dot W_P$                 & Independent space-time white noise processes & $\dot W_H(\pmb x),\dot W_P(\pmb x)$ random variables in $\mathbb R$ \\ \\
$\bar Z_H(\pmb x),\bar Z_P(\pmb x)$ & Local mean traits at $\pmb x$                & $\bar Z_H(\pmb x),\bar Z_P(\pmb x)$ random variables in $\mathbb R$ \\ \\
$\bar z_H(\pmb x),\bar z_P(\pmb x)$ & Expected local mean traits at $\pmb x$       & $\bar z_H(\pmb x),\bar z_P(\pmb x)\in\mathbb R$                     \\ \\ \hline
\end{tabular}
\bigskip{}
\end{table}

\hypertarget{extensions}{%
\subsection{Extensions}\label{extensions}}

\hypertarget{heterogeneous-aboitic-optima}{%
\subsubsection{Heterogeneous aboitic
optima}\label{heterogeneous-aboitic-optima}}

\begin{itemize}
\tightlist
\item
  linear clines, eg:
  \(\theta_H(\pmb x)=\alpha_H+\beta_{H,1}x_1+\beta_{H,2}x_2\)
\item
  quadratic clines, eg:
  \(\theta_H(\pmb x)=\alpha_H+\beta_{H,1}x_1+\beta_{H,2}x_2+\gamma_{H,11}x_1^2+\gamma_{H,22}x_2^2+\gamma_{H,12}x_1x_2\)
\item
  stochastic fields, eg: \(\theta_H(\pmb x)\) follows a fractional
  Brownian surface, but is fixed in time
\end{itemize}

\hypertarget{methods}{%
\section{Methods}\label{methods}}

We use results connecting theory of SPDE, random fields and geostats to
calculate characteristic spatial scales at which various phenomena
become visible.

\hypertarget{discussion}{%
\section{Discussion}\label{discussion}}

\newpage

\begin{center}
  \Large Appendices
\end{center}

\appendix

\hypertarget{a-stable-equilibrium-of-pde}{%
\section{\texorpdfstring{A Stable Equilibrium of PDE
(\ref{deterministic})}{A Stable Equilibrium of PDE ()}}\label{a-stable-equilibrium-of-pde}}

To find an equilibrium of the deterministic PDE (\ref{deterministic}) we
set the time derivatives of \(\bar z_H\) and \(\bar z_P\) equal to zero.
This leads to following systems of ODE's

\begin{subequations}
  \begin{equation}
    0=G_HA_H(\theta_H-\bar z_H)-G_HB_H(\bar z_P-\bar z_H)+\frac{D_H}{2}\Delta\bar z_H,
  \end{equation}
  \begin{equation}
    0=G_PA_P(\theta_P-\bar z_P)+G_PB_P(\bar z_H-\bar z_P)+\frac{D_P}{2}\Delta\bar z_P.
  \end{equation}
\end{subequations}

Solving this system for the mean traits \(\bar z_H(\pmb x)\) and
\(\bar z_P(\pmb x)\) as spatial functions then returns equilibrium
solutions to PDE (\ref{deterministic}). However, we can start simple by
considering spatially homogeneous solutions. In these cases the spatial
derivatives will return zero (in particular,
\(\Delta\bar z_H=\Delta\bar z_P=0\)). Finding these solutions amounts to
solve the following system of algebraic equations

\begin{subequations}
  \begin{equation}
    0=G_HA_H(\theta_H-\bar z_H)-G_HB_H(\bar z_P-\bar z_H),
  \end{equation}
  \begin{equation}
    0=G_PA_P(\theta_P-\bar z_P)+G_PB_P(\bar z_H-\bar z_P).
  \end{equation}
\end{subequations}

This linear system is uniquely solved by

\begin{subequations}\label{non-spatial-eq}
  \begin{equation}
    \bar z_H(\pmb x)\equiv\frac{(A_P+B_P)A_H\theta_H-B_HA_P\theta_P}{(A_P+B_P)A_H-B_HA_P},
  \end{equation}
  \begin{equation}
    \bar z_P(\pmb x)\equiv\frac{(A_H-B_H)A_P\theta_P+B_PA_H\theta_H}{(A_H-B_H)A_P+B_PA_H},
  \end{equation}
\end{subequations}

where the symbol \(\equiv\) is used in place of \(=\) to emphasize these
functions are constant across space.

To understand whether this equilibrium is stable we perform stability
analysis on the non-spatial model. Notice that system
(\ref{non-spatial}) can be rewritten in matrix notation as
\(\vec{\bar z}=M\vec{\bar z}+\vec b\) where

\begin{subequations}
  \begin{equation}
    \vec{\bar z}=\left(\begin{matrix}
      \bar z_H \\ \bar z_P
    \end{matrix}\right), \ 
    \vec b =\left(\begin{matrix}
      G_HA_H\theta_H \\ G_PA_P\theta_P
    \end{matrix}\right),
  \end{equation}
  \begin{equation}
    M=\left(\begin{matrix}
      G_H(B_H-A_H) & -G_HB_H \\
      G_PB_P & -G_P(A_P+B_P)
    \end{matrix}\right).
  \end{equation}
\end{subequations}

It turns that if the real parts of the eigenvalues of \(M\) are negative
then equilibrium (\ref{non-spatial-eq}) is stable. This condition holds
in general when \(B_H<A_H+(A_P+B_P)G_P/G_H\) and \(B_H<A_H(1+B_P/A_P)\).
A sufficient condition with clear biological intuition is \(B_H<A_H\),
which just means the strength of biotic selection on the host is less
than the strength of abiotic stabilizing selection. In this case abiotic
stabilizing selection prevents the host from escaping via evolution.

\hypertarget{stationary-solutions-of-spde}{%
\section{\texorpdfstring{Stationary Solutions of SPDE
(\ref{spde})}{Stationary Solutions of SPDE ()}}\label{stationary-solutions-of-spde}}

To focus on stationary solutions of the stochastic system (\ref{spde})
we set the time derivatives to zero to obtain the following system

\begin{subequations}\label{stationary-spde}
  \begin{equation}
    G_HA_H(\theta_H-\bar Z_H)-G_HB_H(\bar Z_P-\bar Z_H)+\frac{D_H}{2}\Delta\bar Z_H=\sqrt\frac{G_H}{N_H}\dot W_H,
  \end{equation}
  \begin{equation}
    G_PA_P(\theta_P-\bar Z_P)+G_PB_P(\bar Z_H-\bar Z_P)+\frac{D_P}{2}\Delta\bar Z_P=\sqrt\frac{G_P}{N_P}\dot W_P.
  \end{equation}
\end{subequations}

If we further restrict our focus to the single species case we recover a
SPDE of the form

\begin{equation}\label{simple-form}
  b^2(a-u)+\Delta u=\sigma\dot W.
\end{equation}

In the case of two spatial dimensions, it is known that equation
(\ref{simple-form}) is satisfied by a Gaussian field with a Whittle
covariance function (Whittle 1963; Sigrist, Künsch, and Stahel 2014).
Denoting \(K_1\) the modified Bessel function of the second kind, order
1, the Whittle covariance function is given by
\(C(r)=\frac{\sigma^2r}{2b}K_1(br)\). Hence, we can postulate that
solutions of the stationary system (\ref{stationary-spde}) will have
Whittle cross-covariance matrix-valued functions.

\hypertarget{computing-covariance-matrix-from-spectral-representation}{%
\section{Computing Covariance Matrix from Spectral
Representation}\label{computing-covariance-matrix-from-spectral-representation}}

In the technical note of (Hu et al. 2013), a method to compute
covariance functions of multivariate Gaussian random fields from systems
of SPDE was outlined. The approach begins with a stationary system such
as (\ref{stationary-spde}) and applies a Fourier transform
\(\mathcal F\) to convert derivatives into algebraic expressions. The
Fourier transform acts to switch perspective from the two-dimensional
spatial variable \(\pmb x=(x_1,x_2)\) to a two-dimensional frequency
variable (also referred to as wavenumber) \(\pmb k=(k_1,k_2)\), with
\(k_1,k_2\) being complex numbers. We can provide a spectral
characterization of the system by considering the behavior of a system
across a spectrum of frequencies. Once solving for the quantity of
interest in the spectral representation we can use the inverse Fourier
transform \(\mathcal F^{-1}\) to obtain the associated quantity as a
function of space.

To apply this method to our model, we start by using a change of
variables to obtain an equivalent system where each spatial variable has
zero mean. In particular, we set \(\bar\zeta_H=\bar Z_H-\bar z_H\) and
\(\bar\zeta_P=\bar Z_P-\bar z_p\) where \((\bar z_H,\bar z_P)\) is the
spatially homogeneous equilibrium of the deterministic system reported
in equation (\ref{non-spatial-eq}). Under this change of variables, an
Itô formula can be applied (need to fact-check) to show the stationary
system (\ref{stationary-spde}) becomes

\begin{subequations}
  \begin{equation}
    \left(\frac{D_H}{2}\Delta-G_H(A_H-B_H)\right)\bar\zeta_H-G_HB_H\bar\zeta_P=\sqrt\frac{G_H}{N_H}\dot W_H,
  \end{equation}
  \begin{equation}
    G_PB_P\bar \zeta_H+\left(\frac{D_P}{2}\Delta-G_P(A_P+B_P)\right)\bar\zeta_P=\sqrt\frac{G_P}{N_P}\dot W_P.
  \end{equation}
\end{subequations}

To consolidate notation, we set

\begin{subequations}
  \begin{equation}
    \pmb{\mathscr{L}} = \left(\begin{matrix}
      G_H(A_H-B_H)-\frac{D_H}{2}\Delta & G_HB_H \\ & \\
      -G_PB_P & G_P(A_P+B_P)-\frac{D_P}{2}\Delta
    \end{matrix}\right),
  \end{equation}
  \begin{equation}
    \bar{\pmb{\zeta}} = \left(\begin{matrix}
      \bar\zeta_H \\ \\ \bar\zeta_P
    \end{matrix}\right), \ 
    \pmb{V} = \left(\begin{matrix}
      -\sqrt\frac{G_H}{N_H}\dot W_H \\ \\ 
      -\sqrt\frac{G_P}{N_P}\dot W_P
    \end{matrix}\right).
  \end{equation}
\end{subequations}

Then, the stationary system can equally be written as

\begin{equation}
  \pmb{\mathscr{L}}\bar{\pmb\zeta}=\pmb V.
\end{equation}

Setting
\(\hat\zeta_H=\mathcal{F}[\bar\zeta_H], \ \hat\zeta_P=\mathcal{F}[\bar\zeta_P], \hat V_H=\mathcal{F}[-\sqrt{\frac{G_H}{N_H}}\dot W_H], \ \hat V_P=\mathcal{F}[-\sqrt{\frac{G_P}{N_P}}\dot W_P]\),
we can Fourier transform the whole dang thing to get

\begin{equation}
  \pmb{\mathcal{H}}\hat{\pmb\zeta}=\hat{\pmb V},
\end{equation}

where

\begin{subequations}
  \begin{equation}
    \pmb{\mathscr H}=\left(\begin{matrix}
      G_H(A_H-B_H)+\frac{D_H}{2}\|\pmb k\|^2 & G_HB_H \\ & \\
      -G_PB_P & G_P(A_P+B_P)+\frac{D_P}{2}\|\pmb k\|^2
    \end{matrix}\right),
  \end{equation}
  \begin{equation}
    \hat{\pmb\zeta}=\left(\begin{matrix}
      \hat\zeta_H \\ \\ \hat\zeta_P
    \end{matrix}\right), \\
    \hat{\pmb V}=\left(\begin{matrix}
      \hat V_H \\ \\ \hat V_P
    \end{matrix}\right), \\
  \end{equation}
\end{subequations}

The power spectrum corresponding to \(\bar{\pmb\zeta}\) is defined as
\(\pmb S_{\bar{\pmb\zeta}}=\mathbb E\left[\hat{\pmb\zeta}\hat{\pmb\zeta}^\top\right]\),
where \(\top\) denotes matrix transposition. In particular, this
provides

\begin{equation}
  \pmb S_{\bar{\pmb\zeta}}=\left(\begin{matrix}
    S_{\bar\zeta_{HH}} & S_{\bar\zeta_{HP}} \\ & \\
    S_{\bar\zeta_{PH}} & S_{\bar\zeta_{PP}}
  \end{matrix}\right)=\left(\begin{matrix}
    \mathbb E[\hat\zeta_H\hat\zeta_H] & \mathbb E[\hat\zeta_H\hat\zeta_P] \\ & \\
    \mathbb E[\hat\zeta_H\hat\zeta_P] & \mathbb E[\hat\zeta_P\hat\zeta_P]
  \end{matrix}\right)=\pmb{\mathscr H}^{-1}\pmb S_{\pmb V}\pmb{\mathscr H}^{-H},
\end{equation}

with \(-H\) denoting the inverse of the Hermitian of a matrix and
\(\pmb S_{\pmb V}=\mathbb E\left[\hat{\pmb V}\hat{\pmb V}^H\right]\)
denoting the power spectrum of the noise process \(\pmb V\). Using
properties of Fourier transforms, we compute

\begin{equation}
  \hat{\pmb V}=\left(\begin{matrix}
    \sqrt{G_H/N_H} \\ \\
    \sqrt{G_P/N_P}
  \end{matrix}\right).
\end{equation}

Formally, I'm missing some \(2\pi\)'s. However, since our goal is to
transform back into geographic space (all this Fourier stuff puts us in
frequency space) the missing \(2\pi\)'s should be accounted for. Since
we assume the white noise processes \(\dot W_H\) and \(\dot W_P\) are
independent we have

\begin{equation}
  \pmb S_{\pmb V}=\left(\begin{matrix}
    S_{V_H} & 0 \\ & \\
    0 & S_{V_P}
  \end{matrix}\right)=\left(\begin{matrix}
    G_H/N_H & 0 \\ & \\
    0 & G_P/N_P
  \end{matrix}\right).
\end{equation}

Plugging into \emph{Mathematica} yields

\begin{subequations}
  \begin{equation}
    S_{\bar\zeta_{HH}} = \frac{\frac{G_H}{4N_H}\left(2G_P(A_P+B_P)+D_P\|\pmb k\|^2\right)^2+\frac{G_P}{N_P}B_H^2G_H^2}
    {\left\{-B_HB_PG_HG_P-\frac{1}{4}\left[2G_H(A_H-B_H)+D_H\|\pmb k\|^2\right]\left[2G_P(A_P+B_P)+D_P\|\pmb k\|^2\right]\right\}^2}
  \end{equation}
  \begin{equation}
    S_{\bar\zeta_{PP}} = \frac{\frac{G_P}{4N_P}\left(2G_H(A_H-B_H)+D_H\|\pmb k\|^2\right)^2+\frac{G_H}{N_H}B_P^2G_P^2}
    {\left\{-B_HB_PG_HG_P-\frac{1}{4}\left[2G_H(A_H-B_H)+D_H\|\pmb k\|^2\right]\left[2G_P(A_P+B_P)+D_P\|\pmb k\|^2\right]\right\}^2}
  \end{equation}
  \begin{equation}  
    S_{\bar\zeta_{HP}} = \frac{\frac{G_HG_P}{2}\left[\frac{B_P}{N_H}\left(2G_P(A_P+B_P)+D_P\|\pmb k\|^2\right)+\frac{B_H}{N_P}\left(2G_H(A_H-B_H)+D_H\|\pmb k\|^2\right)\right]}
    {\left\{-B_HB_PG_HG_P-\frac{1}{4}\left[2G_H(A_H-B_H)+D_H\|\pmb k\|^2\right]\left[2G_P(A_P+B_P)+D_P\|\pmb k\|^2\right]\right\}^2}.
  \end{equation}
\end{subequations}

Then, by assuming weak biotic selection so that
\(B_H^2,B_P^2,B_HB_P\approx0\), we get

\begin{subequations}
  \begin{equation}
    S_{\bar\zeta_{HH}}(\pmb k) \approx \frac{4G_H/N_H}{\left[2G_H(A_H-B_H)+D_H\|\pmb k\|^2\right]^2},
  \end{equation}
  \begin{equation}
    S_{\bar\zeta_{PP}}(\pmb k) \approx \frac{4G_P/N_P}{\left[2G_P(A_P+B_P)+D_P\|\pmb k\|^2\right]^2},
  \end{equation}
  \begin{multline}  
    S_{\bar\zeta_{HP}}(\pmb k) \approx 8G_HG_P\left(\frac{B_H/N_P}{\left[2G_H(A_H-B_H)+D_H\|\pmb k\|^2\right]\left[2G_P(A_P+B_P)+D_P\|\pmb k\|^2\right]^2}\right. \\
      \left. + \frac{B_P/N_H}{\left[2G_H(A_H-B_H)+D_H\|\pmb k\|^2\right]^2\left[2G_P(A_P+B_P)+D_P\|\pmb k\|^2\right]}\right).
  \end{multline}
\end{subequations}

It may be better to assume weak coupling in the general SPDE model
(\(b_{12},b_{21}\ll1\)) to get this result since there is an impulse to
distribute the \(B\)'s are continue canceling terms, which makes life
harder since the marginal cov's are no longer Whittle.

Taking the inverse Fourier transform of the power spectra
\(S_{\bar\zeta_{HH}}(\pmb k)\) and \(S_{\bar\zeta_{PP}}(\pmb k)\) yields
the (intraspecific) spatial covariance functions
\(C_{HH}(\pmb x),C_{PP}(\pmb x)\) and the inverse Fourier transform of
\(S_{\bar \zeta_{HP}}\) provides the (interspecific) spatial
cross-covariance function \(C_{HP}(\pmb x)\). The intraspecific
covariance functions are straightforward to calculate, leading to the
Whittle covariance functions:

\begin{subequations}
  \begin{equation}
    C_{HH}(\pmb x) \approx \frac{M(\pmb x|1,\sqrt{2G_H(A_H-B_H)/D_H})}{N_HD_H(A_H-B_H)},
  \end{equation}
  \begin{equation}
    C_{PP}(\pmb x) \approx \frac{M(\pmb x|1,\sqrt{2G_P(A_P+B_P)/D_P})}{N_PD_P(A_P+B_P)},
  \end{equation}
\end{subequations}

where
\(M(\pmb x|\nu,\kappa)=\frac{2^{1-\nu}}{\Gamma(\nu)}(\kappa\|\pmb x\|)^\nu K_\nu(\kappa\|\pmb x\|)\)
is the Matern spatial correlation function.

The interspecific cross covariance function does not lend to a simple
closed form expression. Instead, we can see the corresponding component
of the power spectrum matrix is the sum of convolutions of Matern
correlation functions with Bessel functions. In particular, denoting
\(f*g\) the convolution of two functions \(f(\pmb x),g(\pmb x)\), we
have

\begin{multline}
  C_{HP}(\pmb x) \approx \frac{2G_PB_P}{N_HD_HD_P(A_H-B_H)}\left[K_0\left(\sqrt{\frac{2G_P}{D_P}(A_P+B_P)}\|\pmb x\|\right) * M\left(\pmb x\Big|1,\sqrt{\frac{2G_H}{D_H}(A_H-B_H)}\right)\right] \\ - \frac{2G_HB_H}{N_PD_HD_P(A_P+B_P)}\left[K_0\left(\sqrt{\frac{2G_H}{D_H}(A_H-B_H)}\|\pmb x\|\right)*M\left(\pmb x\Big|1,\sqrt{\frac{2G_P}{D_P}(A_P+B_P)}\right)\right].
\end{multline}

Although characteristic length scales are obvious for \(C_{HH}\) and
\(C_{PP}\) (by normalizing distance, they are
\(\xi_H=\sqrt{D_H/[2G_H(A_H-B_H)]}\) and
\(\xi_P=\sqrt{D_P/[2G_P(A_P+B_P)]}\) respectively), finding a
corresponding characteristic length for \(C_{HP}\) is not so obvious.
Fortunately there exist alternative measures of spatial scale that we
can employ. In particular, we can use a measure called \emph{integral
range} to study spatial patterns at a more global scale and a measure
called \emph{smoothness microscale} to study the more fine-grained
details of spatial patterns.

The \emph{integral range} a stationary isotropic process on
\(\mathbb R^2\) characterized by a correlation function \(\rho(\pmb x)\)
is given by

\[l=\sqrt{\int_{\mathbb R^2}\rho(\pmb x)d\pmb x}.\]

However, since the correlation function we are interested may become
negative, we use the modified integral range
\(l'=\sqrt{\left|\int_{\mathbb R^2}\rho(\pmb x)d\pmb x\right|}\).

The \emph{smoothness microscale} has a more complex definition,
involving the Laplacian of the covariance function evaluated at zero lag
in the denominator. This is a potential issue since the Laplacian of
\(K_0(\pmb x)\) and \(M(\pmb x|\kappa,1)\) evaluated at zero lag both
take infinite values.

The \emph{correlation length} may be a feasible alternative, but what
does it tell us that integral range doesn't already cover?

\hypertarget{computing-integral-lengths-for-general-dispersal-kernels}{%
\subsection{Computing Integral Lengths for General Dispersal
Kernels}\label{computing-integral-lengths-for-general-dispersal-kernels}}

For more general dispersal kernels, including heavy-tailed kernels, we
focus on the more general form of the SPDE:

\begin{subequations}
  \begin{equation}
    b_H(\kappa_H^2-\Delta)^{\alpha_H/2}\bar Z_H + b_{HP}\bar Z_P = \sqrt{V_H}\dot W_H
  \end{equation}
  \begin{equation}
    b_{PH}\bar Z_H + b_P(\kappa_P^2-\Delta)^{\alpha_P/2}\bar Z_P = \sqrt{V_P}\dot W_P
  \end{equation}
\end{subequations}

The parameters \(\alpha_H\) and \(\alpha_P\) are determined by the
dispersal kernel. In particular, under diffusive dispersal (ie.,
Gaussian kernels) we have \(\alpha_H=\alpha_P=2\). In general we can
have \(1<\alpha_H,\alpha_P\leq2\) where \(\alpha_H,\alpha_P<2\) implies
heavy-tailed dispersal kernels (stable distributions in particular).

Under our model of coevolution with diffusive dispersal we have the
parameterization

\begin{itemize}
\tightlist
\item
  \(b_H=(D_H/2)^{\alpha_H/2}\),
\item
  \(b_P=(D_P/2)^{\alpha_P/2}\),
\item
  \(b_{HP}=G_HB_H\),
\item
  \(b_{PH}=-G_PB_P\),
\item
  \(\kappa_H^2=2G_H(A_H-B_H)/D_H\),
\item
  \(\kappa_P^2=2G_P(A_P+B_P)/D_P\),
\item
  \(V_H=G_H/N_H\)
\item
  \(V_P=G_P/N_P\)
\end{itemize}

with \(\alpha_H=\alpha_P=2\). An important question is to understand
whether the parameterization of \(\kappa_H\) and \(\kappa_P\) will
change when dispersal follows heavy-tailed distributions so that
\(\alpha_H,\alpha_P<2\). For now we side-step this question by
proceeding with the parameterization using \(b\)'s, \(\kappa\)'s,
\(\alpha\)'s and \(V\)'s.

Keeping our assumption of weakly coupled stochastic fields so that
\(b_{HP}^2,b_{PH}^2,b_{HP}b_{PH}\approx0\), we obtain

\begin{subequations}
  \begin{equation}
    S_{\bar\zeta_{HH}}(\pmb k) \approx \frac{V_H/b_H^2}{(\kappa_H^2+\|\pmb k\|)^{\alpha_H}},
  \end{equation}
  \begin{equation}
    S_{\bar\zeta_{PP}}(\pmb k) \approx \frac{V_P/b_P^2}{(\kappa_P^2+\|\pmb k\|)^{\alpha_P}},
  \end{equation}
  \begin{equation}  
    S_{\bar\zeta_{HP}}(\pmb k) \approx -\frac{b_{PH}V_H}{b_Pb_H^2}(\kappa_P^2+\|\pmb k\|)^{-\alpha_P/2}(\kappa_H^2+\|\pmb k\|)^{-\alpha_H}
      -\frac{b_{HP}V_P}{b_Hb_P^2}(\kappa_H^2+\|\pmb k\|)^{-\alpha_H/2}(\kappa_P^2+\|\pmb k\|)^{-\alpha_P}.
  \end{equation}
\end{subequations}

Using properties of Fourier transforms, we know the local variance of
trait values for each species is given by

\begin{subequations}
  \begin{equation}
    C_{HH}(\pmb 0) = \int_{\mathbb R^2}S_{\bar\zeta_{HH}}(\pmb k)d\pmb k \approx \frac{V_H/b_H^2}{(\alpha_H-1)\kappa_H^{2(\alpha_H-1)}},
  \end{equation}
  \begin{equation}
    C_{PP}(\pmb 0) = \int_{\mathbb R^2}S_{\bar\zeta_{PP}}(\pmb k)d\pmb k \approx \frac{V_P/b_P^2}{(\alpha_P-1)\kappa_P^{2(\alpha_P-1)}}.
  \end{equation}
\end{subequations}

Then, the integral ranges for spatial patterns of intraspecific traits
are

\begin{subequations}
  \begin{equation}
    l_H=\sqrt{\frac{S_{\bar\zeta_{HH}}(\pmb 0)}{C_{HH}(\pmb 0)}}\approx\frac{\alpha_H-1}{\kappa_H}=(\alpha_H-1)\sqrt\frac{D_H}{2G_H(A_H-B_H)},
  \end{equation}
  \begin{equation}
    l_P=\sqrt{\frac{S_{\bar\zeta_{PP}}(\pmb 0)}{C_{PP}(\pmb 0)}}\approx\frac{\alpha_P-1}{\kappa_P}=(\alpha_P-1)\sqrt\frac{D_P}{2G_P(A_P+B_P)}.
  \end{equation}
\end{subequations}

Unfortunately, although the integral
\(\int_{\mathbb R^2}S_{\bar\zeta_{HP}}(\pmb k)d\pmb k\) does have a
closed-form expression, it is quite long and filled with many bizarre
transcendental functions. Hence, the expression for integral range of
interspecific trait patterns may not provide much intuition for the
general case. However, in the case of Gaussian dispersal, we have

\begin{multline}
  C_{HP}(\pmb 0)=\int_{\mathbb R^2}S_{\bar\zeta_{HP}}(\pmb k)d\pmb k \\
    =\frac{b_{PH}V_H}{b_Pb_H^2}\frac{\kappa_H^2(1-\ln\kappa_H^2)-\kappa_P^2(1-\ln\kappa_P^2)}{\kappa_H^2(\kappa_H-\kappa_P)^2(\kappa_H+\kappa_P)^2}
      +\frac{b_{HP}V_P}{b_Hb_P^2}\frac{\kappa_P^2(1-\ln\kappa_P^2)-\kappa_H^2(1-\ln\kappa_H^2)}{\kappa_H^2(\kappa_H-\kappa_P)^2(\kappa_H+\kappa_P)^2}
\end{multline}

Hence, the integral range is slightly easier to read in this case:

\begin{multline}
  l_{HP} = \sqrt{\frac{S_{\bar\zeta_{HP}}(\pmb 0)}{C_{HP}(\pmb 0)}} \\
    =\frac{(\kappa_H^2-\kappa_P^2)^2(b_Pb_{PH}\kappa_P^2V_H+b_Hb_{HP}\kappa_H^2V_P)}{b_Pb_{PH}V_H\kappa_H^2\kappa_P^4(\kappa_P^2-\kappa_H^2(1+\ln(\kappa_P^2/\kappa_H^2)))+b_Hb_{HP}V_P\kappa_P^2\kappa_H^4(\kappa_H^2-\kappa_P^2(1+\ln(\kappa_H^2/\kappa_P^2)))}.
\end{multline}

Although it is possible this expression will greatly simplify when model
parameters are substituted in, it is also possible it will get even
worse. Hence, we might want to search for other ways to quantify
characteristic lengths.

\textbf{Regardless of the metric chosen, we can at least visualize how
the characteristic lengths change as functions of biological parameters,
even for heavy-tailed dispersal kernels.}

\hypertarget{measures-of-local-adaptation}{%
\section{Measures of Local
Adaptation}\label{measures-of-local-adaptation}}

To measure local adaptation as a function of spatial lag \(\pmb x\), we
consider the fitness difference for individuals interacting with local
partners versus partners located at the spatial lag \(\pmb x\). In
particular, denoting \(\bar m_H(\pmb x,\pmb y)\) the growth rate of the
host from location \(\pmb x\) when confronted with a parasite from
location \(\pmb y\) and \(\bar m_P(\pmb y,\pmb x)\) the growth rate of
the parasite from location \(\pmb y\) when confronted with a host from
location \(\pmb x\), we have

\begin{subequations}
  \begin{equation}
    \bar m_H (\pmb x,\pmb y) = r_H - \frac{A_H}{2}(\theta_H-\bar Z_H(\pmb x))^2 + \frac{B_H}{2}(\bar Z_P(\pmb y)-\bar Z_H(\pmb x))^2 - \frac{A_H-B_H}{2}\sigma_H^2 + \frac{B_H}{2}\sigma_P^2,
  \end{equation}
  \begin{equation}
    \bar m_P (\pmb y,\pmb x) = r_P - \frac{A_P}{2}(\theta_P-\bar Z_P(\pmb y))^2 - \frac{B_P}{2}(\bar Z_H(\pmb x)-\bar Z_P(\pmb y))^2 - \frac{A_P+B_P}{2}\sigma_P^2 - \frac{B_P}{2}\sigma_H^2,
  \end{equation}
\end{subequations}

Then, since we consider stationary processes, the fitness difference for
confronting individuals at the lag \(\pmb x\) is given by

\begin{subequations}
  \begin{equation}
    \Delta_H(\pmb x) = \bar m_H (\pmb 0,\pmb 0) - \bar m_H (\pmb 0,\pmb x) = \frac{B_H}{2}\left[(\bar Z_P(\pmb 0)-\bar Z_H(\pmb 0))^2 - (\bar Z_P(\pmb x)-\bar Z_H(\pmb 0))^2\right],
  \end{equation}
  \begin{equation}
    \Delta_P(\pmb x) = \bar m_P (\pmb 0,\pmb 0) - \bar m_P (\pmb 0,\pmb x) = \frac{B_P}{2}\left[(\bar Z_H(\pmb x)-\bar Z_P(\pmb 0))^2 - (\bar Z_H(\pmb 0)-\bar Z_P(\pmb 0))^2\right].
  \end{equation}
\end{subequations}

Taking expectations provides

\begin{subequations}
  \begin{multline}
    \mathbb E[\Delta_H(\pmb x)] = \\ \frac{B_H}{2}\left[(\bar z_P-\bar z_H)^2 + C_{PP}(\pmb 0) + C_{HH}(\pmb 0) - 2C_{HP}(\pmb 0) - (\bar z_P-\bar z_H)^2 - C_{PP}(\pmb 0) - C_{HH}(\pmb 0) + 2C_{HP}(\pmb x) \right],
  \end{multline}
  \begin{multline}
    \mathbb E[\Delta_P(\pmb x)] = \\ \frac{B_P}{2}\left[(\bar z_H-\bar z_P)^2 + C_{PP}(\pmb 0) + C_{HH}(\pmb 0) - 2C_{HP}(\pmb x) - (\bar z_H-\bar z_P)^2 - C_{PP}(\pmb 0) - C_{HH}(\pmb 0) + 2C_{HP}(\pmb 0) \right].
  \end{multline}
\end{subequations}

Since \(\bar z_P,\bar z_H\) are spatially homogeneous, this simplifies
to

\begin{subequations}
  \begin{equation}
    \mathbb E[\Delta_H(\pmb x)] = B_H\left[C_{HP}(\pmb x) - C_{HP}(\pmb 0) \right],
  \end{equation}
  \begin{equation}
    \mathbb E[\Delta_P(\pmb x)] = B_P\left[C_{HP}(\pmb 0) - C_{HP}(\pmb x) \right].
  \end{equation}
\end{subequations}
\newpage

\hypertarget{environmental-heterogeneity}{%
\section{Environmental
Heterogeneity}\label{environmental-heterogeneity}}

We can extend our model to allow the abiotic optimum, representing
environmental condition, to vary across space. We now denote
\(\Theta_H(\pmb x),\Theta_P(\pmb x)\) the random fields corresponding to
the abiotic optimum at location \(\pmb x\) for the host and parasite
respectively. Then, assuming the abiotic optima follow Whittle
covariance functions, we have

\begin{subequations}
  \begin{equation}
    G_HA_H(\Theta_H-\bar Z_H)-G_HB_H(\bar Z_P-\bar Z_H)+\frac{D_H}{2}\Delta\bar Z_H=\sqrt\frac{G_H}{N_H}\dot W_H,
  \end{equation}
  \begin{equation}
    G_PA_P(\Theta_P-\bar Z_P)+G_PB_P(\bar Z_H-\bar Z_P)+\frac{D_P}{2}\Delta\bar Z_P=\sqrt\frac{G_P}{N_P}\dot W_P,
  \end{equation}
  \begin{equation}
    \gamma_H(\theta_H-\Theta_H)+\frac{R_H}{2}\Delta\Theta_H=\dot Q_H,
  \end{equation}
  \begin{equation}
    \gamma_P(\theta_P-\Theta_P)+\frac{R_P}{2}\Delta\Theta_P=\dot Q_P,
  \end{equation}
\end{subequations}

where \(\dot Q_H,\dot Q_P\) are spatial white-noise processes driving
the abiotic optima, \(\gamma_H,\gamma_P\) determine the variance around
the deterministic and spatially homogeneous \(\theta_H,\theta_P\) and
\(R_H,R_P\) determine the spatial covariance (but not cross-covariance)
of the abiotic optima. This model can be further generalized so that
\(\Theta_H(\pmb x),\Theta_P(\pmb x)\) follow general Matern covariance
functions to get

\begin{subequations}
  \begin{equation}
    G_HA_H(\Theta_H-\bar Z_H)-G_HB_H(\bar Z_P-\bar Z_H)+\frac{D_H}{2}\Delta\bar Z_H=\sqrt\frac{G_H}{N_H}\dot W_H,
  \end{equation}
  \begin{equation}
    G_PA_P(\Theta_P-\bar Z_P)+G_PB_P(\bar Z_H-\bar Z_P)+\frac{D_P}{2}\Delta\bar Z_P=\sqrt\frac{G_P}{N_P}\dot W_P,
  \end{equation}
  \begin{equation}
    \gamma_H\theta_H+\left(-\gamma_H+\frac{R_H}{2}\Delta\right)^{\beta_H/2}\Theta_H=\dot Q_H,
  \end{equation}
  \begin{equation}
    \gamma_P\theta_P\left(-\gamma_P+\frac{R_P}{2}\Delta\right)^{\beta_P/2}\Theta_P=\dot Q_P.
  \end{equation}
\end{subequations}

Now \(\beta_H,\beta_P\) determine long-range spatial correlations of the
abiotic optima (with themselves, not each other) along with the
roughness of the fields \(\Theta_H,\Theta_P\).

\hypertarget{references}{%
\section*{References}\label{references}}
\addcontentsline{toc}{section}{References}

\hypertarget{refs}{}
\leavevmode\hypertarget{ref-hu2013multivariate}{}%
Hu, Xiangping, Daniel Simpson, Finn Lindgren, and Håvard Rue. 2013.
``Multivariate Gaussian Random Fields Using Systems of Stochastic
Partial Differential Equations.'' \emph{arXiv Preprint arXiv:1307.1379}.

\leavevmode\hypertarget{ref-Sigrist2014}{}%
Sigrist, Fabio, Hans R. Künsch, and Werner A. Stahel. 2014. ``Stochastic
Partial Differential Equation Based Modelling of Large Space-Time Data
Sets.'' \emph{Journal of the Royal Statistical Society: Series B
(Statistical Methodology)} 77 (1): 3--33.
\url{https://doi.org/10.1111/rssb.12061}.

\leavevmode\hypertarget{ref-whittle1963stochastic}{}%
Whittle, Peter. 1963. ``Stochastic-Processes in Several Dimensions.''
\emph{Bulletin of the International Statistical Institute} 40 (2):
974--94.

\bibliographystyle{unsrt}
\bibliography{references.bib}


\end{document}
